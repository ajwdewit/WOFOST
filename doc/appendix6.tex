% This file was created by the WP2LaTeX program version: 3.51 
\documentclass[11pt]{article}
\usepackage{wp2latex}
\usepackage[USenglish]{babel}


\begin{document}
\setcounter{page}{141}\pagenumpos{\pnbr}
\section{  APPENDIX 6  }

\bigskip
{\bf {\large FORTRAN utility library: TTUTIL}}

This appendix describes a number of FORTRAN subroutines and functions that make up
the FORTRAN utility library TTUTIL. The model uses a number of the utilities offered
by this library. The complete TTUTIL library is available for the user of the model,
therefore, a brief description is included in this manual. For more detailed information
reference is made to Rappoldt, C and D.W.G. van Kraalingen (1991).

Many routines are utilities in the sense that they do not make use of any mathemati\-cal or
numerical method, do not contain measured data and do not depend on assumptions
concerning some described system. Utilities simply perform their task with respect to
input, output, string handling, file handling etc. They are tools for writing reliable and
readable FORTRAN programs. 

Two groups of routines have a numerical character. These are the random number
generators and the routines solving an initial value problem by means of the Runge-Kutta
or Euler method. Further, a few utilities deal with often recurring calculations in
numerical simulation models.

In all subroutines and functions present in TTUTIL a conservative programming style is
used. Nested constructions in single statements are omitted (they are not very clear
anyway) and known compiler deficiencies are taken into account. All subrou\-tines and
functions run on an Atari ST computer with the Prospero compiler, on an IBM PC with
the Microsoft compiler, on an Apple MacIntosh with the Absoft compiler and on the
VAX using the VAX-FORTRAN compiler. A few changes have to be made before using
the library on an Atari ST.

\bigskip
{\bf The structure of the library TTUTIL}

Table A3 gives a classification of the TTUTIL routines. The first eight groups list
generally applicable subroutines and functions. The next four groups are related to
simulation models, to a smaller or larger extent. The last group consists of two simple
routines for manipulating a VT100 screen. Closely related routines have names beginning
with the same acronym. For instance, the "DEC" routines DECCHK, DECINT,
DECREA and DECREC are used for decoding character strings into real or integer
values.

\bigskip
\bigskip
\bigskip
\bigskip
\bigskip
\bigskip
  \bigskip
Table A3 
\testlastline

\begin{indenting}{2.54cm}
Classification of the TTUTIL library routines
\end{indenting}
\GrBox(1000)\GrBox(1010)\GrBox(1010)\GrBox(1010)\GrBox(1010)\GrBox(1010)\GrBox(1010)\GrBox(1010)\GrBox(1010)\GrBox(1010)\GrBox(1010)\GrBox(1010)\GrBox(1010)\GrBox(1010)\GrBox(1010)\GrBox(1010)\GrBox(1010)\GrBox(1010)\GrBox(1010)\GrBox(1010)\GrBox(1010)\GrBox(1010)\GrBox(1010)\GrBox(1010)\GrBox(1010)\GrBox(1010)\GrBox(1010)\GrBox(1010)\GrBox(1010)\GrBox(1010)\GrBox(1010)\GrBox(1010)\GrBox(1010)\GrBox(1010)\GrBox(1010)\GrBox(1010)\GrBox(1010)\-\\
{\bf Interactive input}
\testlastline

\begin{indenting}{6.35cm}
ENTREA, ENTINT, ENTCHA, ENTDRE, ENT\-DIN, ENTD\-CH 
\end{indenting}
{\bf Input from file}
\testlastline

\begin{indenting}{6.35cm}
RDAREA, RDDATA, RDFROM, RDINDX, RDINIT, RDSCHA,
RDSE\-TS, RDSINT, RDSRE\-A, GETCH, GETREC, MOFILP, COPFIL
\end{indenting}
{\bf Output to file}
\testlastline

\begin{indenting}{6.35cm}
OUTCOM, OUTDAT, OUTPLT, OUTARR, PLTFUN, PLTHIS,
COPFIL
\end{indenting}
{\bf Character string handling}
\testlastline

\begin{indenting}{6.35cm}
DECCHK, EXTENS, IFINDC, ILEN, ISTART, REMOVE, STRIP,
UPPE\-RC, WORDS
\end{indenting}
\begin{tabbing}
\hspace{1.27cm}\=\hspace{1.27cm}\=\hspace{1.27cm}\=\hspace{1.27cm}\=%
\hspace{1.27cm}\=\hspace{1.27cm}\=\hspace{1.27cm}\=\hspace{1.27cm}\=%
\hspace{1.27cm}\=\hspace{1.27cm}\=\kill
{\bf Decoding character strings to values}\> \> \> \> \> DECCHK, DECINT, DECREA, DECREC
\end{tabbing}
{\bf Files}
\testlastline

\begin{indenting}{6.35cm}
EXTENS, FOPEN, FOPENG, COPFIL
\end{indenting}
{\bf Error handling}
\testlastline

\begin{indenting}{6.35cm}
ERROR
\end{indenting}
{\bf AFGEN tables (lists of X,Y pairs)}
\testlastline

\begin{indenting}{6.35cm}
AFINVS, LINT, PLTFUN, PLTHIS
\end{indenting}
\begin{tabbing}
\hspace{1.27cm}\=\hspace{1.27cm}\=\hspace{1.27cm}\=\hspace{1.27cm}\=%
\hspace{1.27cm}\=\hspace{1.27cm}\=\hspace{1.27cm}\=\hspace{1.27cm}\=%
\hspace{1.27cm}\=\hspace{1.27cm}\=\kill
{\bf Random numbers}\> \> \> \> \> BOXMUL, GAMMA, UNIFL
\end{tabbing}
{\bf Euler method in (crop) simulation}
\testlastline

\begin{indenting}{6.35cm}
CHKTSK, INTGRL, TIMER
\end{indenting}
{\bf Euler and Runge-Kutta method}
\testlastline

\begin{indenting}{6.35cm}
EUDRIV, RKDRIV, RKQCA, RK4A
\end{indenting}
\begin{tabbing}
\hspace{1.27cm}\=\hspace{1.27cm}\=\hspace{1.27cm}\=\hspace{1.27cm}\=%
\hspace{1.27cm}\=\hspace{1.27cm}\=\hspace{1.27cm}\=\hspace{1.27cm}\=%
\hspace{1.27cm}\=\hspace{1.27cm}\=\kill
{\bf Emulation of some CSMP functions}\> \> \> \> \> INSW, LIMIT, LINT, REAAND, REA\-NOR
\end{tabbing}
{\bf VT100 screen}
\testlastline

\begin{indenting}{6.35cm}
CLS, POS
\end{indenting}
\GrBox(1000)\GrBox(1010)\GrBox(1010)\GrBox(1010)\GrBox(1010)\GrBox(1010)\GrBox(1010)\GrBox(1010)\GrBox(1010)\GrBox(1010)\GrBox(1010)\GrBox(1010)\GrBox(1010)\GrBox(1010)\GrBox(1010)\GrBox(1010)\GrBox(1010)\GrBox(1010)\GrBox(1010)\GrBox(1010)\GrBox(1010)\GrBox(1010)\GrBox(1010)\GrBox(1010)\GrBox(1010)\GrBox(1010)\GrBox(1010)\GrBox(1010)\GrBox(1010)\GrBox(1010)\GrBox(1010)\GrBox(1010)\GrBox(1010)\GrBox(1010)\GrBox(1010)\GrBox(1010)\GrBox(1010)\-

Table A3 can be used for an efficient search through the library if you are looking for a
routine that solves a specific programming problem. 

\bigskip
\bigskip
{\bf The ENT routines}

The usual way to obtain interactive input from the user is to write a question to the screen
and to read the answer from the screen. Exactly that is the function of the simple routines
ENTCHA, ENTINT and ENTREA. They can be used to ask for a character string, an
integer value and a real value, respectively. For instance, the statement:

\hspace*{1.28cm}CALL ENTREA ('Size of square',SIZE)

writes the question "Size of square" to the screen and the number returned is assigned to
the real variable SIZE. Several such calls together form a relatively short program section
for interactive input. Successive questions are written neatly below each other and the
cursor is always in column 53 of the screen, independent of question length.

Somewhat less trivial are the subroutines ENTDCH, ENTDIN and ENTDRE. Again, the
three routines are meant for entering a character string, an integer value or a real value,
respectively. As an additional input argument, however, they accept a default value. The
default value is returned to the calling program when the user does not type in a new
value and presses the $<$Enter$>$ key only. The three ENTD routines write the default
value between square brackets following the question. For instance, the statement:

\hspace*{1.28cm}CALL ENTDRE ('Size of square',2.300,SIZE)

\bigskip
\bigskip
causes the following line being written to  the screen:

\hspace*{1.28cm}Size of square [2.3]:

The user either supplies a new value or just presses $<$Return$>$ to accept the default.
Note that the second argument (the default value) may also be a variable. The variable
SIZE could be used, for instance, as the second and  third argument of ENTDRE. Than
the (current) value of SIZE is used as the default answer.

In order to keep the cursor behind the question, the ENT routines make use of a "\$" in
the output format string. This is not allowed in standard FORTRAN and it forms the only
exception to the use of standard FORTRAN in the entire library. There seems to be no
standard way, however, to keep the cursor after writing to the screen. An additional
advan\-tage of using the ENT routines is that interactive input sections can be kept free of a
non-standard dollar.

\bigskip
\bigskip
{\bf The RD routines}

The ordinary method for reading data from a file consists of a number of READ
statements, each reading data from a record of the file. That method clearly requires that
the sequence of READ statements is consistent with the contents of the file. Moreover,
array lengths have to be known in the program or have to be read as separate data items.
Formatted reading, moreover, requires accurate positioning of the data items. In general,
much time is invested in debugging such "simple" input sections.

The solution suggested sometimes in text books on FORTRAN is to read data as character
strings and to perform the decoding in the program. That requires a considerable
program\-ming effort and a need was felt for generally applicable input routines based on
that principle. The RD routines enable the construction of clear, short and robust input
sections consisting of CALL's only. The idea is that the input file contains both the
variable name \uline{and} the associated value(s). Then, the RD routines are able to find the
value(s) to be assigned to a certain input variable. The syntax rules for the data files are
the following:

$\bullet$
\testlastline

\begin{indenting}{1.27cm}
The file consists of variable names and numerical values, separated by an '='
sign;
\end{indenting}
$\bullet$
\testlastline

\begin{indenting}{1.27cm}
Variable names have a maximum length of six characters;
\end{indenting}
$\bullet$
\testlastline

\begin{indenting}{1.27cm}
More than one value may appear on the right-hand side of the equal sign, separat\-ed by commas.
\end{indenting}

\zerotestlastline
\begin{indenting}{1.27cm}
This indicates that the variable is an array. In case of conti\-{\nobreak}nua\-tion on a following
line the comma should be on the preceding line;
\end{indenting}
$\bullet$
\testlastline

\begin{indenting}{1.27cm}
Values may be repeated. For instance, 100 times 1.0 can be written as 100*1.0;
\end{indenting}
$\bullet$
\testlastline

\begin{indenting}{1.27cm}
Variables may appear in any order in the file;
\end{indenting}
$\bullet$
\testlastline

\begin{indenting}{1.27cm}
Comment lines start with "*" in the first column, or "!" in any column (rest of
line is ignored);
\end{indenting}

 \bigskip
$\bullet$
\testlastline

\begin{indenting}{1.27cm}
Different name--{}value combinations on the same line should be separated by a
{\nobreak}semi--{}colon ";";
\end{indenting}
$\bullet$
\testlastline

\begin{indenting}{1.27cm}
Only the first 80 characters of each line of the data file are decoded;
\end{indenting}
$\bullet$
\testlastline

\begin{indenting}{1.27cm}
The decimal point in a real value like 20. may be omitted;
\end{indenting}
$\bullet$
\testlastline

\begin{indenting}{1.27cm}
No TAB's may occur.
\end{indenting}

 \bigskip
The subroutines RDINIT, RDSREA, RDAREA and RDSINT are just user interfaces to a
lower level subroutine RDDATA which is called by these four. Sometimes the name
RDDATA occurs in error messages, for instance when a variable name is not present in
the file or in case of an array length error. In such error messages also the variable name
is given. The suspect RD call is then easily identified.

Using the RD routines, one should be aware of a few limitations. All values that occur in
a data file are stored in an unformatted temporary file as single precision REAL numbers
(4 byte REAL's). That implies that the number of significant digits is about 6. To obtain
an integer number, routine RDSINT simply performs a nearest  integer operation on the
stored value (using the intrinsic NINT function). In case of integer values larger than
about 1,000,000 the result may differ from the contents of the data file. Future versions
may require the use of a decimal point in real numbers. Hence, you may save yourself
some (future) error messages by using these in all real numbers in your files. Double
precision numbers will probably not be supported. Instructions will be added, however,
for adapting the whole set of RD routines for reading double precision numbers.

\bigskip
\bigskip
{\bf Making reruns with the RD routines}

The use of the RD routines for input has the additional advantage of a built in "rerun
facility". Calculations often need to be carried out for different values of input variables.
Suppose that something is calculated using the input variables BB, EE and FF from the
above file INPUT.DAT. The calculations have to be repeated for different combinations
of BB and EE. Then a so called rerun file can be created by the user, containing the
desired combinations of BB and EE.

A rerun file consists of sets of variables. The order of the variables should be identical in
all sets. The syntax rules are identical to those of an ordinary data file. Actual array
lengths may differ between sets.

This requires some further explanation: With a call to FOPEN a log file is opened, which
is used for writing a report on rerun file usage. In the call to RDSETS the rerun file is
analyzed and a short report is written to the log file (unit 40). When the rerun file is not
present or empty the output variable INSETS is set to zero. Other\-wise the number of
variable sets is returned. By means of the call to RDFROM in the DO--{}loop, a certain set
is selected. Selection of set zero means that the contents of the original data file will be
used. The input section for reading the values of BB, EE and FF is just an ordinary input
section for reading variables from a data file. The RD routines, however, internally check
whether reruns are being made and whether a non--{}zero set was selected. In that case, for
variables occurring in the rerun file, the data file contents are replaced by the contents of
the rerun file. Since this is a rather hidden activity, each replacement is reported to the
log file. The CLOSE statements remove the temporary files created by the RD routines.

The file "RERUNS.DAT" may be absent or present. If the file is absent, the above
program section will carry out only one run using the contents of the data file.

The rerun facility has a global character, i.e. variables stored in different data files may
occur in a single rerun file. Within the above DO--{}loop, for instance, the variables BB
and EE could be read from different input files by writing two separate input sections
(each containing a call to RDINIT). As a consequence, the use of identical variable names
in different input files leads to problems when reruns are made for that variable. Then the
value of both  variables will be replaced by the contents of the rerun file. Both replace\-ments will be reported to the produced log file.
\end{document}
