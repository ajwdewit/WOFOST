\section{  APPENDIX 7  }

\bigskip
{\bf {\large Description of Common Blocks}}

\bigskip
CROPDO
\testlastline

\begin{indenting}{2.54cm}
WLV1, WST1, WSO1, LAI1, DVS1, TSUM, TRA1, GASS1, MRES1, DMI1, TAGP1, IDOST,
IDWST, IDOSJ, IDWSJ
\end{indenting}
\begin{tabbing}
\hspace{1.27cm}\=\hspace{1.27cm}\=\hspace{1.27cm}\=\hspace{1.27cm}\=%
\hspace{1.27cm}\=\hspace{1.27cm}\=\hspace{1.27cm}\=\hspace{1.27cm}\=%
\hspace{1.27cm}\=\hspace{1.27cm}\=\kill
\>\> Purpose\> : Daily output of crop variables from CROPSI to WOFOST.OUT.\\
\>\> In\> : CROPSI/CRSIM
\end{tabbing}

\bigskip
CROPFO
\testlastline

\begin{indenting}{2.54cm}
IDANTX, IDHALT, TWRTX, TWLVX, TWSTX, TWSOX, TAGPX, GASSTX, MRESTX, HINDXX,
TRCX, TRATX
\end{indenting}
\begin{tabbing}
\hspace{1.27cm}\=\hspace{1.27cm}\=\hspace{1.27cm}\=\hspace{1.27cm}\=%
\hspace{1.27cm}\=\hspace{1.27cm}\=\hspace{1.27cm}\=\hspace{1.27cm}\=%
\hspace{1.27cm}\=\hspace{1.27cm}\=\kill
\>\> Purpose\> : Final output of crop variables from CROPSI to WOFOST.OUT.\\
\>\> In\> : CROPSI/CRSIM
\end{tabbing}

\nwln
\begin{tabbing}
\hspace{1.27cm}\=\hspace{1.27cm}\=\hspace{1.27cm}\=\hspace{1.27cm}\=%
\hspace{1.27cm}\=\hspace{1.27cm}\=\hspace{1.27cm}\=\hspace{1.27cm}\=%
\hspace{1.27cm}\=\hspace{1.27cm}\=\kill
CRPSI\>  YLVPP, YSTPP, YSOPP, HIPP, RATPP, IDFLPP, DURPP, TRATPP, TRCPP, YRTPP, YAGPP,
\end{tabbing}
GASPP, RESPP, LAMXPP, YLVWL, YSTWL, YSOWL, HIWL, RATWL, IDFLWL, DURWL,
TRATWL, TRCWL, YRTWL, YAGWL, GASWL, RESWL, LAMXWL, IDWET, IDDRY
\hspace*{1.28cm}
\testlastline

\begin{indenting}{2.54cm}
Purpose
\end{indenting}
\begin{indenting}{3.81cm}
: Final results from subroutine CROPSI. In the general version of WOFOST these variables 
  needed for yield statistics, calculation of nutrient limited production.
\end{indenting}
\begin{tabbing}
\hspace{1.27cm}\=\hspace{1.27cm}\=\hspace{1.27cm}\=\hspace{1.27cm}\=%
\hspace{1.27cm}\=\hspace{1.27cm}\=\hspace{1.27cm}\=\hspace{1.27cm}\=%
\hspace{1.27cm}\=\hspace{1.27cm}\=\kill
\>\> In : CROPSI/CRSIM STATPP, STATWP.
\end{tabbing}

\bigskip
SUBTOT
\testlastline

\begin{indenting}{2.54cm}
TRAJ, EVSJ, EVWJ, RAINJ, RINJ, RIRRJ, DWJ, PERCJ, LOSSJ, DWLOWJ, WWLOWJ, CRJ,
DMAXJ, DZJ
\end{indenting}
\begin{tabbing}
\hspace{1.27cm}\=\hspace{1.27cm}\=\hspace{1.27cm}\=\hspace{1.27cm}\=%
\hspace{1.27cm}\=\hspace{1.27cm}\=\hspace{1.27cm}\=\hspace{1.27cm}\=%
\hspace{1.27cm}\=\hspace{1.27cm}\=\kill
\>\> Purpose\> : Subtotals water balance variables (set to zero after printing).\\
\>\> In\> : WATFD
\end{tabbing}

\nwln
\begin{tabbing}
\hspace{1.27cm}\=\hspace{1.27cm}\=\hspace{1.27cm}\=\hspace{1.27cm}\=%
\hspace{1.27cm}\=\hspace{1.27cm}\=\hspace{1.27cm}\=\hspace{1.27cm}\=%
\hspace{1.27cm}\=\hspace{1.27cm}\=\kill
SUBTOT2\> \> TRAJWL, RELSM\\
(JRC version)\> \> Purpose\> : Subtotals water balance variables.\\
\>\> In\> : WATFD
\end{tabbing}

\nwln
\begin{tabbing}
\hspace{1.27cm}\=\hspace{1.27cm}\=\hspace{1.27cm}\=\hspace{1.27cm}\=%
\hspace{1.27cm}\=\hspace{1.27cm}\=\hspace{1.27cm}\=\hspace{1.27cm}\=%
\hspace{1.27cm}\=\hspace{1.27cm}\=\kill
SUBTOT3\> \> TRAJPP\\
(JRC version)\> \> Purpose\> : Subtotal potential transpiration, potential production\\
\>\> In\> : WATPP
\end{tabbing}

\nwln
\begin{tabbing}
\hspace{1.27cm}\=\hspace{1.27cm}\=\hspace{1.27cm}\=\hspace{1.27cm}\=%
\hspace{1.27cm}\=\hspace{1.27cm}\=\hspace{1.27cm}\=\hspace{1.27cm}\=%
\hspace{1.27cm}\=\hspace{1.27cm}\=\kill
WATPPO\> \> TRATX, EVWTX, EVSTX\\
\>\> Purpose\> : Output water balance variables from subroutine WATPP\\
\>\> In\> : WATPP, STATPP
\end{tabbing}

\bigskip
WBALFD
\testlastline

\begin{indenting}{2.54cm}
TRATX, EVWTX, EVSTX, TSRX, RAINTX, WDRTX, TINFX, TIRRX, PERCTX, SSIX, SSX,
WIX, WX, WLOWIX, WLOWX, WBRTX, WBTOTX, LOSSTX, MWCX, TWEX
\end{indenting}
\begin{tabbing}
\hspace{1.27cm}\=\hspace{1.27cm}\=\hspace{1.27cm}\=\hspace{1.27cm}\=%
\hspace{1.27cm}\=\hspace{1.27cm}\=\hspace{1.27cm}\=\hspace{1.27cm}\=%
\hspace{1.27cm}\=\hspace{1.27cm}\=\kill
\>\> Purpose : output variables for summary water balance without ground water influence in\\
 \>\> \>   WOFOST.OUT\\
\>\> In\> : WATFD, STATWP
\end{tabbing}

\bigskip
WBALGW
\testlastline

\begin{indenting}{2.54cm}
TRATX, EVWTX, EVSTX, TSRX, RAINTX, WDRTX, TINFX, TIRRX, PERCTX, DELW, DELSS,
WIX, WX, WBRTX, WBTOTX, CRTX, DRAITX, DELWZ
\end{indenting}
\begin{tabbing}
\hspace{1.27cm}\=\hspace{1.27cm}\=\hspace{1.27cm}\=\hspace{1.27cm}\=%
\hspace{1.27cm}\=\hspace{1.27cm}\=\hspace{1.27cm}\=\hspace{1.27cm}\=%
\hspace{1.27cm}\=\hspace{1.27cm}\=\kill
\>\> Purpose : output variables for summary water balance with ground water influence in\\
 \>\> \>   WOFOST.OUT\\
\>\> In\> : WATGW, STATWP
\end{tabbing}

\begin{tabbing}
\hspace{1.27cm}\=\hspace{1.27cm}\=\hspace{1.27cm}\=\hspace{1.27cm}\=%
\hspace{1.27cm}\=\hspace{1.27cm}\=\hspace{1.27cm}\=\hspace{1.27cm}\=%
\hspace{1.27cm}\=\hspace{1.27cm}\=\kill
 WFDDO\> \> RAINT1, EVW1, EVS1, SM1, SS1, WWLOW1\\
\>\> Purpose\> : daily output of water variables from WATFD\\
\>\> In\> : WATFD
\end{tabbing}

\nwln
\begin{tabbing}
\hspace{1.27cm}\=\hspace{1.27cm}\=\hspace{1.27cm}\=\hspace{1.27cm}\=%
\hspace{1.27cm}\=\hspace{1.27cm}\=\hspace{1.27cm}\=\hspace{1.27cm}\=%
\hspace{1.27cm}\=\hspace{1.27cm}\=\kill
WGWDO\> \> RAINT1, EVW1, EVS1, SM1, SS1, ZT1\\
\>\> Purpose\> : daily output of water variables from WATGW\\
\>\> In\> : WATGW
\end{tabbing}

\bigskip
WRROUT
\testlastline

\begin{indenting}{2.54cm}
MYLVPP, MYSTPP, MYSOPP, VYCPP, MYLVWP, MYSTWP, MYSOWP, VCYWP, MYLVXP,
MYSTXP, MYSOXP, VCYXP
\end{indenting}
\begin{tabbing}
\hspace{1.27cm}\=\hspace{1.27cm}\=\hspace{1.27cm}\=\hspace{1.27cm}\=%
\hspace{1.27cm}\=\hspace{1.27cm}\=\hspace{1.27cm}\=\hspace{1.27cm}\=%
\hspace{1.27cm}\=\hspace{1.27cm}\=\kill
\>\> Purpose\> : writing of output\\
\>\> In\> : STATPP, STATWP
\end{tabbing}

\bigskip
(Note: STATPP and STATWP are not activated in the JRC version of WOFOST 6.0)

\bigskip
\bigskip
\bigskip
\bigskip
{\bf {\large Acronyms}}

\nwln
\begin{tabbing}
\hspace{1.27cm}\=\hspace{1.27cm}\=\hspace{1.27cm}\=\hspace{1.27cm}\=%
\hspace{1.27cm}\=\hspace{1.27cm}\=\hspace{1.27cm}\=\hspace{1.27cm}\=%
\hspace{1.27cm}\=\hspace{1.27cm}\=\kill
{\bf SUBROUTINE ASSIM}\> \> \> (AMAX, EFF, LAI, KDIF, SINB, PARDIR, PARDIF, FGROS)
\end{tabbing}
\nwln
\begin{tabbing}
\hspace{1.27cm}\=\hspace{1.27cm}\=\hspace{1.27cm}\=\hspace{1.27cm}\=%
\hspace{1.27cm}\=\hspace{1.27cm}\=\hspace{1.27cm}\=\hspace{1.27cm}\=%
\hspace{1.27cm}\=\hspace{1.27cm}\=\kill
Name \> \> Type \> Description                                       \> \> \> \> \> \> \> Units\\
-$-$$-$$-$ \> \> $-$$-$$-$$-$\> $-$$-$$-$$-$$-$$-$$-$$-$$-$$-$$-$                                        \> \> \> \> \> \> \> $-$$-$$-$$-$$-$ \\
AMAX\> \> R   \> maximum leaf CO$_{{\rm 2}}$ assimilation rate       \> \> \> \> \> \> \> kg ha$^{{\rm -1}}$ h$^{{\rm -1}}$ \\
EFF \> \> R   \> initial light$-$use efficiency of CO$_{{\rm 2}}$ assimilation of single leaves  \> \> \> \> \> \> \> kg ha$^{{\rm -1}}$ h$^{{\rm -1}}$  J$^{{\rm -1}}$ m$^{{\rm 2}}$\\
FGL \> \> R   \> gross CO$_{{\rm 2}}$ ass. rate at a certain depth and LAIC in the crop canopy\> \> \> \> \> \> \> kg ha$^{{\rm -1}}$ h$^{{\rm -1}}$ \\
FGROS\> \> R   \> gross CO$_{{\rm 2}}$ assimilation rate per hour of the canopy \> \> \> \> \> \> \> kg ha$^{{\rm -1}}$ h$^{{\rm -1}}$ \\
FGRSH\> \> R   \> gross CO$_{{\rm 2}}$ ass. rate of shaded leaves at a certain depth in canopy \> \> \> \> \> \> \> kg ha$^{{\rm -1}}$ h$^{{\rm -1}}$ \\
FGRSUN\> \> R   \> gross CO$_{{\rm 2}}$ ass. rate of sunlit leaves at a certain depth in canopy  \> \> \> \> \> \> \> kg ha$^{{\rm -1}}$ h$^{{\rm -1}}$ \\
FSLLA   \> \> R   \> fraction of sunlit leaf area                 \> \> \> \> \> \> \> $-$\\
I       \> \> I   \> DO$-$loop control variable                    \> \> \> \> \> \> \> $-$\\
KDIF    \> \> R   \> extinction coefficient for diffuse visible light\> \> \> \> \> \> \> $-$\\
KDIRBL  \> \> R   \> extinction coefficient for the direct component of direct radiation\> \> \> \> \> \> \> $-$\\
KDIRT   \> \> R   \> extinction coefficient for total direct radiation     \> \> \> \> \> \> \> $-$ \\
LAI     \> \> R \> leaf area index                                    \> \> \> \> \> \> \> ha ha$^{{\rm -1}}$\\
LAIC    \> \> R   \> leaf area index indicating different depths in the canopy \> \> \> \> \> \> \> ha ha$^{{\rm -1}}$\\
PARDIF  \> \> R   \> flux of diffuse photosynthetically active radiation \> \> \> \> \> \> \> J m$^{{\rm -2}}$ s$^{{\rm -1}}$ \\
PARDIR  \> \> R   \> flux of direct photosynthetically active radiation \> \> \> \> \> \> \> J m$^{{\rm -2}}$ s$^{{\rm -1}}$\\
REFH    \> \> R   \> reflection coefficient of canopy with horizontal leaves        \> \> \> \> \> \> \> $-$\\
REFS    \> \> R   \> reflection coefficient of canopy with spherical leaf angle distribution\> \> \> \> \> \> \> $-$\\
SCV     \> \> R   \> scattering coefficient of a leaf                \> \> \> \> \> \> \> $-$\\
SINB    \> \> R   \> sine of solar elevation                        \> \> \> \> \> \> \> $-$\\
VISD    \> \> R   \> absorbed flux of direct component of direct rad.\> \> \> \> \> \> \> J m$^{{\rm -2}}$ s$^{{\rm -1}}$ \\
VISDF   \> \> R   \> absorbed flux of diffuse radiation              \> \> \> \> \> \> \> J m$^{{\rm -2}}$ s$^{{\rm -1}}$ \\
VISPP   \> \> R   \> absorbed flux of direct rad. by leaves perpendicular to direct beam\> \> \> \> \> \> \> J m$^{{\rm -2}}$ s$^{{\rm -1}}$ \\
VISSHD  \> \> R   \> absorbed flux by shaded leaves       \> \> \> \> \> \> \> J m$^{{\rm -2}}$ s$^{{\rm -1}}$ \\
VIST    \> \> R   \> absorbed flux of total direct radiation \> \> \> \> \> \> \> J m$^{{\rm -2}}$ s$^{{\rm -1}}$ \\
WGAUSS    \> \> R   \> mathematical constants used for three$-$point Gaussian integration  \> \> \> \> \> \> \> $-$\\
XGAUSS    \> \> R   \> mathematical constants used for three$-$point Gaussian integration   \> \> \> \> \> \> \> $-$
\end{tabbing}

\bigskip
\bigskip
\bigskip
\nwln
\begin{tabbing}
\hspace{1.27cm}\=\hspace{1.27cm}\=\hspace{1.27cm}\=\hspace{1.27cm}\=%
\hspace{1.27cm}\=\hspace{1.27cm}\=\hspace{1.27cm}\=\hspace{1.27cm}\=%
\hspace{1.27cm}\=\hspace{1.27cm}\=\kill
{\bf SUBROUTINE ASTRO}\> \> \> (IDAY, LAT, AVRAD, DAYL, DAYL, DAYLP, \\
\>\> \>  SINLD, COSLD, DIFPP, ATMTR, DSINBE)
\end{tabbing}
\nwln
\begin{tabbing}
\hspace{1.27cm}\=\hspace{1.27cm}\=\hspace{1.27cm}\=\hspace{1.27cm}\=%
\hspace{1.27cm}\=\hspace{1.27cm}\=\hspace{1.27cm}\=\hspace{1.27cm}\=%
\hspace{1.27cm}\=\hspace{1.27cm}\=\kill
Name    \> \> Type\> Description                    \> \> \> \> \>    \> \> Units\\
-$-$$-$$-$    \> \> $-$$-$$-$$-$\> $-$$-$$-$$-$$-$$-$$-$$-$$-$$-$$-$                \> \> \>           \> \> \> \> $-$$-$$-$$-$$-$\\
ANGLE   \> \> R   \> angle of the sun                          \> \> \> \> \> \> \> degrees\\
ANGOT\> \> R\> extra-terrestrial radiation\> \> \> \> \> \> \> J m$^{{\rm -2}}$ d$^{{\rm -1}}$\\
AOB     \> \> R\> auxiliary variable                            \> \> \> \> \> \> \> $-$\\
ATMTR   \> \> R   \> fraction of Angot's radiation actually received    \> \> \> \> \> \> \> $-$\\
ATMTRA\> \> R\> array of calculated daily ATMTR values\> \> \> \> \> \> \> -\\
AVRAD\> \> R\> daily shortwave radiation actually received\> \> \> \> \> \> \> J m$^{{\rm -2}}$ d$^{{\rm -1}}$\\
AVRADA\> \> R\> array of calculated daily AVRAD values\> \> \> \> \> \> \> J m$^{{\rm -2}}$ d$^{{\rm -1}}$\\
COSLD   \> \> R  \> cosine(latitude) times cosine(declination of sun)\> \> \> \> \> \> \> $-$\\
COSLDA\> \> R\> array of calculated daily COSLD values\> \> \> \> \> \> \> -\\
DAYL \> \> R \> day length                              \> \> \> \> \> \> \> h\\
DAYLA\> \> R\> array of calculated daily DAYL values\> \> \> \> \> \> \> h\\
DAYLP   \> \> R \> photoperiodic day length                 \> \> \> \> \> \> \> h\\
DAYLPA\> \> R\> array of calculated daily DAYLP values\> \> \> \> \> \> \> h\\
DEC     \> \> R \> declination of the sun                    \> \> \> \> \> \> \> radians\\
DECA\> \> R\> array of calculated daily DEC values\> \> \> \> \> \> \> radians\\
DIFPP\> \> R\> diffuse radiation perpendicular to direction of light\> \> \> \> \> \> \> J m$^{{\rm -2}}$ s$^{{\rm -1}}$\\
DIFPPA\> \> R\> array of calculated daily DIFPP values\> \> \> \> \> \> \> J m$^{{\rm -2}}$ s$^{{\rm -1}}$\\
DSINB   \> \> R   \> integral of sine of solar elevation over the day               \> \> \> \> \> \> \> s d$^{{\rm -1}}$  \\
DSINBE  \> \> R   \> integral of sine of solar elevation over the day corrected for \\
\>\> \> lower atmospheric transmission at low solar elevations               \> \> \> \> \> \> \> s d$^{{\rm -1}}$\\
DSNBE\> \> R  \> array of calculated daily DSINBE values\> \> \> \> \> \> \> s d$^{{\rm -1}}$\\
FRDIF   \> \> R   \> diffuse radiation as fraction of shortwave radiation actually received\> \> \> \> \> \> \> $-$\\
IDAY    \> \> I  \> day number in the Julian calendar (1 to 365)\> \> \> \> \> \> \> $-$\\
LAT     \> \> R   \> geographical latitude of location        \> \> \> \> \> \> \> degrees\\
LATA\> \> R\> array of daily LAT values\> \> \> \> \> \> \> degrees\\
PI      \> \> R   \> $\pi$\> \> \> \> \> \> \> -\\
 RAD     \> \> R   \> factor to convert degrees to radians       \> \> \> \> \> \> \> rad degree$^{{\rm -1}}$\\
SC      \> \> R   \> solar constant                                     \> \> \> \> \> \> \> J m$^{{\rm -2}}$ s$^{{\rm -1}}$\\
SCA\> \> R\> array of calculated daily SC values\> \> \> \> \> \> \> J m$^{{\rm -2}}$ s$^{{\rm -1}}$\\
SINLD   \> \> R   \> sin(latitude) times sine(declination of sun) \> \> \> \> \> \> \> $-$\\
SINLDA\> \> R\> array of calculated daily SINLD values\> \> \> \> \> \> \> -
\end{tabbing}

\bigskip
\bigskip
\bigskip
\nwln
\begin{tabbing}
\hspace{1.27cm}\=\hspace{1.27cm}\=\hspace{1.27cm}\=\hspace{1.27cm}\=%
\hspace{1.27cm}\=\hspace{1.27cm}\=\hspace{1.27cm}\=\hspace{1.27cm}\=%
\hspace{1.27cm}\=\hspace{1.27cm}\=\kill
{\bf SUBROUTINE CRSIM}\> \> \> (ITASK, IDAY, DELT, TIME, IDEM, DOANTH, IDHALT, TERMNL, ISTATE,\\
(JRC version) \> \> \>  IWB, IOX, LAT, AVRAD, TMIN, TMAX, EO, ESO, ETO, CROP\_NO, VAR\_NO,\\
\>\> \>  SM, SMO, SMFCF, SMW, CRAIRC, EVWMX, EVSMX,\\
 \>\> \>  TRA, FR, RRI, IAIRDU, RDI, RDMCR, CGM\_ABORT)
\end{tabbing}

\nwln
\begin{tabbing}
\hspace{1.27cm}\=\hspace{1.27cm}\=\hspace{1.27cm}\=\hspace{1.27cm}\=%
\hspace{1.27cm}\=\hspace{1.27cm}\=\hspace{1.27cm}\=\hspace{1.27cm}\=%
\hspace{1.27cm}\=\hspace{1.27cm}\=\kill
{\bf SUBROUTINE CROPSI}\> \> \> (ITASK, IDAY, DELT, TIME, IDEM, DOANTH, IDHALT, TERMNL, ISTATE,\\
(General version)\> \> \>  IWB, IOX, LAT, AVRAD, TMIN, TMAX, EO, ESO, ETO, CRFILE, IUPL,\\
\>\> \>  IUOUT, IULOG, SM, SMO, SMFCF, SMW, CRAIRC, EVWMX, EVSMX,\\
 \>\> \>  TRA, FR, RRI, IAIRDU, RDI, RDMCR)
\end{tabbing}
\nwln
\begin{tabbing}
\hspace{1.27cm}\=\hspace{1.27cm}\=\hspace{1.27cm}\=\hspace{1.27cm}\=%
\hspace{1.27cm}\=\hspace{1.27cm}\=\hspace{1.27cm}\=\hspace{1.27cm}\=%
\hspace{1.27cm}\=\hspace{1.27cm}\=\kill
Name    \> \> Type   \> Description                                        \> \> \> \> \> \> \> Units\\
-$-$$-$$-$    \> \> $-$$-$$-$$-$   \> $-$$-$$-$$-$$-$$-$$-$$-$$-$$-$$-$                                        \> \> \> \> \> \> \> $-$$-$$-$$-$$-$ \\
ADMI    \> \> R   \> above$-$ground dry$-$matter increase                   \> \> \> \> \> \> \> kg ha$^{{\rm -1}}$ d$^{{\rm -1}}$\\
AMAX    \> \> R   \> maximum leaf CO$_{{\rm 2}}$ assimilation rate                 \> \> \> \> \> \> \> kg ha$^{{\rm -1}}$ h$^{{\rm -1}}$\\
AMAXTB\> \> R\> AFGEN table with maximum CO$_{{\rm 2}}$ assimilation rate as a function of\\
\>\> \> development stage of the crop (AFGEN table)\> \> \> \> \> \> \> kg ha$^{{\rm -1}}$ h$^{{\rm -1}}$\\
ASRC    \> \> R   \> carbohydrates available for dry matter increase    \> \> \> \> \> \> \> kg ha$^{{\rm -1}}$ h$^{{\rm -1}}$\\
ATMTR\> \> R\> daily atmospheric transmission\> \> \> \> \> \> \> -\\
AVRAD   \> \> R   \> shortwave radiation actually received              \> \> \> \> \> \> \> J m$^{{\rm -2}}$ d$^{{\rm -1}}$\\
CCHECK  \> \> R   \> check on carbon balance                            \> \> \> \> \> \> \> $-$\\
CFET\> \> R\> correction factor for evapotranspiration\> \> \> \> \> \> \> -\\
CGM\_ABORT\> \> L\> error message on database handling\> \> \> \> \> \> \> -\\
COSLD   \> \> R   \> cosine(latitude) times cosine(declination of the sun)             \> \> \> \> \> \> \> -\\
CRAIRC\> \> R\> critical air content in the root zone\> \> \> \> \> \> \> cm$^{{\rm 3}}$ cm$^{{\rm -3}}$\\
CRFILE\> \> C\> name of input file with crop data\> \> \> \> \> \> \> - \\
CROP\_NO\> \> I\> crop number\> \> \> \> \> \> \> -\\
CVF     \> \> R   \> average conversion efficiency of assimilates into crop dry matter  \> \> \> \> \> \> \> kg kg$^{{\rm -1}}$\\
CVL\> \> R\> efficiency of conversion of assimilates into leaf dry matter\> \> \> \> \> \> \> kg kg$^{{\rm -1}}$\\
CVO\> \> R\> efficiency of conversion of assimilates into storage organ dry matter\> \> \> \> \> \> \> kg kg$^{{\rm -1}}$\\
CVR\> \> R\> efficiency of conversion of assimilates into root dry matter\> \> \> \> \> \> \> kg kg$^{{\rm -1}}$\\
CVS\> \> R\> efficiency of conversion of assimilates into stem dry matter\> \> \> \> \> \> \> kg kg$^{{\rm -1}}$\\
DALV    \> \> R   \> amount of leaves dying during current time step as result of ageing\> \> \> \> \> \> \> kg ha$^{{\rm -1}}$\\
DAYL    \> \> R   \> day length                                         \> \> \> \> \> \> \> h\\
DAYLP   \> \> R   \> photoperiodic day length                           \> \> \> \> \> \> \> h\\
DELT    \> \> R   \> time step = 1 day                                  \> \> \> \> \> \> \> d\\
DEPNR\> \> R\> crop group number\> \> \> \> \> \> \> -\\
DIFPP\> \> R\> diffuse irradiation perpendicular to the direction of light\> \> \> \> \> \> \> J m$^{{\rm -2}}$ s$^{{\rm -1}}$\\
DLC\> \> R\> critical day length for development (lower threshold)\> \> \> \> \> \> \> h\\
DLO\> \> R\> optimum daylength for development\> \> \> \> \> \> \> h\\
DMI     \> \> R   \> rate of dry$-$matter increase of the crop            \> \> \> \> \> \> \> kg ha$^{{\rm -1}}$ d$^{{\rm -1}}$\\
DMI1\> \> R\> rate of dry$-$matter increase of the crop (output variable)\> \> \> \> \> \> \> kg ha$^{{\rm -1}}$ d$^{{\rm -1}}$\\
DOANTH\> \> I\> day of anthesis\> \> \> \> \> \> \> - \\
DRLV    \> \> R   \> total death rate of leaves                         \> \> \> \> \> \> \> kg ha$^{{\rm -1}}$ d$^{{\rm -1}}$\\
DRRT    \> \> R   \> death rate of roots                                \> \> \> \> \> \> \> kg ha$^{{\rm -1}}$ d$^{{\rm -1}}$\\
DRSO    \> \> R   \> death rate of storage organs                       \> \> \> \> \> \> \> kg ha$^{{\rm -1}}$ d$^{{\rm -1}}$\\
DRST    \> \> R   \> death rate of stems                                \> \> \> \> \> \> \> kg ha$^{{\rm -1}}$ d$^{{\rm -1}}$\\
DSINBE\> \> R\> integral over the sine of solar elevation with\\
\>\> \> a correction for lower atmospheric transmission\> \> \> \> \> \> \> s\\
DSLV    \> \> R   \> amount of dead leaves due to water stress or due to high LAI \> \> \> \> \> \> \> kg ha$^{{\rm -1}}$ d$^{{\rm -1}}$\\
DSLV1   \> \> R   \> potential death rate of leaves due to water stress \> \> \> \> \> \> \> kg ha$^{{\rm -1}}$ d$^{{\rm -1}}$\\
DSLV2   \> \> R   \> potential death rate of leaves due to high LAI     \> \> \> \> \> \> \> kg ha$^{{\rm -1}}$ d$^{{\rm -1}}$\\
DSLVT   \> \> R   \> amount of dead leaves used to calculate last remaining class \\
\>\> \> of living leaves \> \> \> \> \> \> \> ha$^{{\rm -1}}$\\
DTEFF\> \> R\> effective temperature\> \> \> \> \> \> \> \degrees C\\
DTEMP   \> \> R   \> average day temperature                            \> \> \> \> \> \> \> \degrees C\\
DTGA    \> \> R   \> gross CO$_{{\rm 2}}$ assimilation rate                       \> \> \> \> \> \> \> kg ha$^{{\rm -1}}$ d$^{{\rm -1}}$\\
DTSMTB\> \> R\> daily increase in temperature sum as a function \\
\>\> \> of temperature (AFGEN table)\> \> \> \> \> \> \> \degrees C\\
DTSUM\> \> R\> daily increase in temperature sum\> \> \> \> \> \> \> \degrees C\\
DTSUME\> \> R\> daily increase in temperature sum at emergence\> \> \> \> \> \> \> \degrees C\\
DURPP\> \> R\> duration crop growth (potential production)\> \> \> \> \> \> \> d\\
DURWL\> \> R\> duration crop growth (water limited production)\> \> \> \> \> \> \> d\\
DVR     \> \> R   \> actual development rate of the crop                \> \> \> \> \> \> \> d$^{{\rm -1}}$\\
DVRED   \> \> R   \> reduction factor for development rate, function of day length  \> \> \> \> \> \> \> $-$\\
DVS     \> \> R   \> development stage of the crop                      \> \> \> \> \> \> \> $-$\\
DVS1\> \> R\> development stage of the crop (output variable)              \> \> \> \> \> \> \> $-$\\
DVSEND\> \> R\> development stage at harvest\> \> \> \> \> \> \> -\\
DWLV    \> \> R   \> dry weight of dead leaves                          \> \> \> \> \> \> \> kg ha$^{{\rm -1}}$\\
DWRT    \> \> R   \> dry weight of dead roots                           \> \> \> \> \> \> \> kg ha$^{{\rm -1}}$\\
DWSO    \> \> R   \> dry weight of dead storage organs                  \> \> \> \> \> \> \> kg ha$^{{\rm -1}}$\\
DWST    \> \> R   \> dry weight of dead stems                           \> \> \> \> \> \> \> kg ha$^{{\rm -1}}$\\
E0      \> \> R   \> potential evaporation rate from water surface      \> \> \> \> \> \> \> cm d$^{{\rm -1}}$\\
EFF\> \> R\> initial light use efficiency of CO$_{{\rm 2}}$ assimilation of single leaves \> \> \> \> \> \> \> kg ha$^{{\rm -1}}$ h$^{{\rm -1}}$ J$^{{\rm -1}}$ m$^{{\rm 2}}$s\\
ES0     \> \> R   \> potential evaporation rate from soil surface      \> \> \> \> \> \>  \> cm d$^{{\rm -1}}$\\
ET0     \> \> R   \> potential evapotranspiration rate                  \> \> \> \> \> \> \> cm d$^{{\rm -1}}$\\
EVSMX\> \> R\> maximum evaporation rate from soil surface\> \> \> \> \> \> \> cm d$^{{\rm -1}}$\\
EVWMX\> \> R\> maximum evaporation rate from water surface\> \> \> \> \> \> \> cm d$^{{\rm -1}}$\\
FCHECK  \> \> R   \> check on partitioning fractions                    \> \> \> \> \> \> \> $-$\\
FL      \> \> R   \> fraction of shoot dry$-$matter increase partitioned to the leaves\> \> \> \> \> \> \> - 
\end{tabbing}

\begin{tabbing}
\hspace{1.27cm}\=\hspace{1.27cm}\=\hspace{1.27cm}\=\hspace{1.27cm}\=%
\hspace{1.27cm}\=\hspace{1.27cm}\=\hspace{1.27cm}\=\hspace{1.27cm}\=%
\hspace{1.27cm}\=\hspace{1.27cm}\=\kill
FLTB\> \> R\> fraction of above-ground dry-matter increase partitioned to leaves\\
\>\> \> as a function of development stage (AFGEN table)\> \> \> \> \> \> \> -\\
FO      \> \> R   \> fraction of shoot dry$-$matter increase partitioned to storage organs\> \> \> \> \> \> \> $-$\\
FOTB\> \> R\> fraction of above-ground dry-matter increase partitioned to storage\\
\>\> \> organs as a function of development stage (AFGEN table)\> \> \> \> \> \> \> -\\
FR      \> \> R   \> fraction of dry$-$matter increase partitioned to the roots      \> \> \> \> \> \> \> $-$\\
FRTB\> \> R\> fraction of total dry-matter increase partitioned to roots\\
\>\> \> as a function of development stage (AFGEN table)\> \> \> \> \> \> \> -\\
FS      \> \> R   \> fraction of shoot dry$-$matter increase partitioned to the stems \> \> \> \> \> \> \> $-$\\
FSTB\> \> R\> fraction of above-ground dry-matter increase partitioned to stems\\
\>\> \> as a function of development stage (AFGEN table)\> \> \> \> \> \> \> -\\
FYSDEL  \> \> R   \> physiologic reduction factor for leave age increase            \> \> \> \> \> \> \> $-$\\
GASS\> \> R\> actual gross assimilation rate of the canopy (output variable)\> \> \> \> \> \> \> kg ha$^{{\rm -1}}$ d$^{{\rm -1}}$\\
GASPP\> \> R\> total gross assimilation rate of the canopy, potential production  \> \> \> \> \> \> \> kg ha$^{{\rm -1}}$ d$^{{\rm -1}}$\\
GASWL\> \> R\> total gross assimilation rate of the canopy, water limited production  \> \> \> \> \> \> \> kg ha$^{{\rm -1}}$ d$^{{\rm -1}}$\\
GASS1   \> \> R   \> actual gross assimilation rate of the canopy \> \> \> \> \> \> \> kg ha$^{{\rm -1}}$ d$^{{\rm -1}}$\\
GASST\> \> R\> total gross assimilation rate of the canopy \> \> \> \> \> \> \> kg ha$^{{\rm -1}}$ d$^{{\rm -1}}$\\
GASSTX\> \> R\> total gross assimilation rate of the canopy (output variable) \> \> \> \> \> \> \> kg ha$^{{\rm -1}}$ d$^{{\rm -1}}$\\
GLA\> \> R\> net growth rate of leaf area index\> \> \> \> \> \> \> ha leaf ha$^{{\rm -1}}$ d$^{{\rm -1}}$\\
GLAIEX\> \> R\> exponential growth rate of leaf area index\> \> \> \> \> \> \> ha leaf ha$^{{\rm -1}}$ d$^{{\rm -1}}$\\
GLASOL\> \> R\> source limited growth rate of leaf area index\> \> \> \> \> \> \> ha leaf ha$^{{\rm -1}}$ d$^{{\rm -1}}$\\
GRLV    \> \> R   \> rate of increase of leaf dry matter          \> \> \> \> \> \> \> kg ha$^{{\rm -1}}$ d$^{{\rm -1}}$\\
GRRT    \> \> R   \> rate of increase of root dry matter                \> \> \> \> \> \> \> kg ha$^{{\rm -1}}$ d$^{{\rm -1}}$\\
GRST    \> \> R   \> rate of increase of stem dry matter                \> \> \> \> \> \> \> kg ha$^{{\rm -1}}$ d$^{{\rm -1}}$\\
GWRT    \> \> R   \> rate of increase of living root dry matter         \> \> \> \> \> \> \> kg ha$^{{\rm -1}}$ d$^{{\rm -1}}$\\
GWSO    \> \> R   \> rate of increase of storage organ dry matter       \> \> \> \> \> \> \> kg ha$^{{\rm -1}}$ d$^{{\rm -1}}$\\
GWST    \> \> R   \> rate of increase of living stem dry matter         \> \> \> \> \> \> \> kg ha$^{{\rm -1}}$ d$^{{\rm -1}}$\\
HINDEX\> \> R\> harvest index\> \> \> \> \> \> \> -\\
HINDXX\> \> R\> harvest index (output variable)\> \> \> \> \> \> \> -\\
HIPP\> \> R\> harvest index potential production (output variable)\> \> \> \> \> \> \> -\\
HIWL\> \> R\> harvest index water limited production (output variable)\\
I1\> \> I\> DO-loop control variable\> \> \> \> \> \> \> -\\
I2      \> \> I   \> auxiliary variable                                 \> \> \> \> \> \> \> $-$\\
IAIRDU\> \> I\> indicates presence (1) or absence (0) of airducts in the plant\> \> \> \> \> \> \> -\\
IDANTH\> \> I\> date of anthesis\> \> \> \> \> \> \> -\\
IDANTX\> \> I\> date of anthesis (output variable)\> \> \> \> \> \> \> -\\
IDAY    \> \> I   \> day number in the Julian calendar (between 1 and 365)        \> \> \> \> \> \> \> -\\
IDDRY\> \> I\> total of days with water deficit\> \> \> \> \> \> \> d\\
IDEM\> \> I\> day of emergence\> \> \> \> \> \> \> -\\
IDFLPP\> \> I\> date of anthesis potential production (output variable)\> \> \> \> \> \> \> -\\
IDFLWL\> \> I\> date of anthesis water limited production (output variable)\> \> \> \> \> \> \> -\\
IDHALT\> \> I\> day number at which the simulation is halted\> \> \> \> \> \> \> -\\
IDHALX\> \> I\> day number at which the simulation is halted (summary of results)\> \> \> \> \> \> \> -\\
IDOS\> \> I\> indicates a day with waterlogging (1) or absence of waterlogging (0)\> \> \> \> \> \> \> -\\
IDOSJ\> \> I\> subtotal (over print interval) of number of days with waterlogging\> \> \> \> \> \> \> d\\
IDOST\> \> I\> total of days with waterlogging\> \> \> \> \> \> \> -\\
IDSL\> \> I\> indicates whether pre-anthesis development depends on \\
\>\> \> (0) temperature, (1) daylength or (2) temperature and daylength\> \> \> \> \> \> \> -\\
IDWET\> \> I\> total of days with waterlogging (output variable)\> \> \> \> \> \> \> d\\
IDWS\> \> I\> indicates a day with water deficit (1) or absence of water deficit (0)\> \> \> \> \> \> \> -\\
IDWSJ\> \> I\> subtotal (over print interval) of number of days with water deficit\> \> \> \> \> \> \> d\\
IDWST\> \> I\> total of days with water deficit\> \> \> \> \> \> \> d\\
ILAMAX\> \> I\> number of elements of table AMAXTB\> \> \> \> \> \> \> -\\
ILDTSM\> \> I\> number of elements of table DTSMTB\> \> \> \> \> \> \> -\\
ILFL\> \> I\> number of elements of table FLTB\> \> \> \> \> \> \> -\\
ILFO\> \> I\> number of elements of table FOTB\> \> \> \> \> \> \> -\\
ILFR\> \> I\> number of elements of table FRTB\> \> \> \> \> \> \> -\\
ILFS\> \> I\> number of elements of table FSTB\> \> \> \> \> \> \> -\\
ILRDRR\> \> I\> number of elements of table RDRRTB\> \> \> \> \> \> \> -\\
ILRDRS\> \> I\> number of elements of table RDRSTB\> \> \> \> \> \> \> -\\
ILRFSE\> \> I\> number of elements of table RFSETB\> \> \> \> \> \> \> -\\
ILSLA\> \> I\> number of elements of table SLATB\> \> \> \> \> \> \> -\\
ILTMNF\> \> I\> number of elements of table TMNFTB\> \> \> \> \> \> \> -\\
ILTMPF\> \> I\> number of elements of table TMPFTB\> \> \> \> \> \> \> -\\
ILVOLD  \> \> I   \> class number of the oldest living leaves           \> \> \> \> \> \> \> -\\
ISTATE\> \> I\> state indicator of the model\> \> \> \> \> \> \> -\\
\>\> \> (0) model starts 10 days prior to earliest possible sowing date\\
\>\> \> (1) model starts at sowing\\
\>\> \> (3) model starts at emergence\\
ITASK\> \> I\> task which the subroutine should perform\> \> \> \> \> \> \> -\\
\>\> \> ITASK = 1, initialization\\
\>\> \> ITASK = 2, rate calculation\\
\>\> \> ITASK = 3, integration\\
\>\> \> ITASK = 4, finish section\\
ITOLD\> \> I\> retains number of last task performed\> \> \> \> \> \> \> -\\
IULOG\> \> I\> unit of log file\> \> \> \> \> \> \> -\\
IUOUT\> \> I\> unit of output file with simulation results\> \> \> \> \> \> \> -\\
IUPL\> \> I \> unit of input file with crop data\> \> \> \> \> \> \> -\\
IWB     \> \> I\> flag; controlling calculation of water limited yield without (0) or \\
\>\> \> with (1) accounting for oxygen shortage in the root zone\> \> \> \> \> \> \> -\\
IOX\> \> I\> flag controlling calculation of yield without (0) or with (1) accounting\\
\>\> \> for oxygen shortage in the root zone\> \> \> \> \> \> \> -\\
KDIF\> \> R\> extinction coefficient for diffuse visible light\> \> \> \> \> \> \> -\\
LAI     \> \> R   \> leaf area index                                    \> \> \> \> \> \> \> ha ha$^{{\rm -1}}$\\
LAI1\> \> R\> leaf area index (output variable)\> \> \> \> \> \> \> ha ha$^{{\rm -1}}$\\
LAICR   \> \> R   \> critical leaf area index                           \> \> \> \> \> \> \> ha ha$^{{\rm -1}}$\\
LAIEM\> \> R\> leaf area index at emergence\> \> \> \> \> \> \> ha ha$^{{\rm -1}}$\\
LAIEXP\> \> R\> leaf area index during the exponential growing stage\> \> \> \> \> \> \> ha ha$^{{\rm -1}}$\\
LAIMAX\> \> R\> maximum leaf area index (death plus living leaves)\> \> \> \> \> \> \> ha ha$^{{\rm -1}}$\\
LAMXPP\> \> R\> maximum leaf area (potential production)\> \> \> \> \> \> \> ha ha$^{{\rm -1}}$\\
LAMXWL\> \> R\> maximum leaf area (water limited production)\> \> \> \> \> \> \> ha ha$^{{\rm -1}}$\\
LASUM   \> \> R   \> sum of leaf areas for all living leaves            \> \> \> \> \> \> \> ha ha$^{{\rm -1}}$\\
LAT\> \> R\> latitude of the site\> \> \> \> \> \> \> degrees\\
LV      \> \> R   \> array with leaf weights for each leaf class        \> \> \> \> \> \> \> kg ha$^{{\rm -1}}$\\
LVAGE   \> \> R   \> array with the physiological age for each leaf class       \> \> \> \> \> \> \> d\\
MRES    \> \> R   \> maintenance respiration rate\> \> \> \> \> \> \> kg CH$_{{\rm 2}}$O ha$^{{\rm -1}}$ d$^{{\rm -1}}$\\
MRES1\> \> R\> maintenance respiration rate (output variable)\> \> \> \> \> \> \> kg CH$_{{\rm 2}}$O ha$^{{\rm -1}}$ d$^{{\rm -1}}$\\
MREST\> \> R\> summation over the time steps of the maintenance respiration\> \> \> \> \> \> \> kg CH$_{{\rm 2}}$O ha$^{{\rm -1}}$ d$^{{\rm -1}}$\\
MRESTX\> \> R\> summation over the time steps of the maintenance respiration\> \> \> \> \> \> \> kg CH$_{{\rm 2}}$O ha$^{{\rm -1}}$ d$^{{\rm -1}}$\\
\>\> \> (summary variable)\\
PERDL\> \> R\> maximum relative death rate of leaves due to water stress\> \> \> \> \> \> \> d$^{{\rm -1}}$\\
PGASS   \> \> R   \> gross assimilation rate of the canopy, expressed in carbohydrates   \> \> \> \> \> \> \> kg ha$^{{\rm -1}}$ d$^{{\rm -1}}$\\
Q10\> \> R\> relative increase of the maintenance respiration rate per 10\degrees C\\
\>\> \> temperatue increase \> \> \> \> \> \> \> -\\
RATIO\> \> R\> grain straw ratio\> \> \> \> \> \> \> -\\
RATPP\> \> R\> grain straw ratio (potential production)\> \> \> \> \> \> \> -\\
RATWL\> \> R\> grain straw ratio (water limited production)\> \> \> \> \> \> \> -\\
RDI\> \> R\> initial rooting depth\> \> \> \> \> \> \> cm\\
RDMCR\> \> R\> crop-dependent maximum rooting depth\> \> \> \> \> \> \> cm\\
RDRRTB\> \> R\> relative death rate of roots as a function of DVS (AFGEN table)\> \> \> \> \> \> \> kg kg$^{{\rm -1}}$ d$^{{\rm -1}}$\\
RDRSTB\> \> R\> relative death rate of stems as a function of DVS (AFGEN table)\> \> \> \> \> \> \> kg kg$^{{\rm -1}}$ d$^{{\rm -1}}$\\
RESPP\> \> R\> summation over the time steps of the maintenance \\
\>\> \> respiration (potential production)\> \> \> \> \> \> \> kg CH$_{{\rm 2}}$O ha$^{{\rm -1}}$ d$^{{\rm -1}}$\\
REST\> \> R\> extra death of leave due to exceedance of life span\> \> \> \> \> \> \> kg ha$^{{\rm -1}}$\\
RESWL\> \> R\> summation over the time steps of the maintenance \\
\>\> \> respiration (water limited production)\> \> \> \> \> \> \> kg CH$_{{\rm 2}}$O ha$^{{\rm -1}}$ d$^{{\rm -1}}$\\
RFSETB\> \> R\> reduction factor of the maintenance respiration as a function \\
\>\> \> of DVS (AFGEN table)\> \> \> \> \> \> \> -\\
RGRLAI\> \> R\> maximum relative increase in leaf area index\> \> \> \> \> \> \> ha ha$^{{\rm -1}}$ d$^{{\rm -1}}$\\
RML\> \> R\> relative maintenance respiration rate of leaves\> \> \> \> \> \> \> d$^{{\rm -1}}$\\
RMO\> \> R\> relative maintenance respiration rate of storage organs\> \> \> \> \> \> \> d$^{{\rm -1}}$\\
RMR\> \> R\> relative maintenance respiration rate of roots\> \> \> \> \> \> \> d$^{{\rm -1}}$ \\
 RMRES   \> \> R   \> maintenance respiration rate of crop at average air \\
\>\> \> temperature of 25 \degrees C \> \> \> \> \> \> \> kg ha$^{{\rm -1}}$ d$^{{\rm -1}}$\\
RMS\> \> R\> relative maintenance respiration rate of stems\> \> \> \> \> \> \> d$^{{\rm -1}}$\\
RRI\> \> R\> maximum daily increase of rooting depth\> \> \> \> \> \> \> cm d$^{{\rm -1}}$\\
SINLD   \> \> R   \> sine(latitude) times sine(declination of the sun)                  \> \> \> \> \> \> \> -\\
SLA     \> \> R   \> array with specific leaf areas for each leaf class       \> \> \> \> \> \> \> ha kg$^{{\rm -1}}$\\
SLAT\> \> R\> specific leaf area index\> \> \> \> \> \> \> ha kg$^{{\rm -1}}$\\
SLATB\> \> R\> specific leaf area as a function of development stage (AFGEN table)\> \> \> \> \> \> \> ha kg$^{{\rm -1}}$\\
SM\> \> R\> soil moisture content in the rooted zone\> \> \> \> \> \> \> cm$^{{\rm 3}}$ cm$^{{\rm -3}}$\\
SMFCF\> \> R\> soil moisture content at field capacity\> \> \> \> \> \> \> cm$^{{\rm 3}}$ cm$^{{\rm -3}}$\\
SMO\> \> R\> soil porosity, saturated moisture content\> \> \> \> \> \> \> cm$^{{\rm 3}}$ cm$^{{\rm -3}}$\\
SMW\> \> R\> soil moisture content at wilting point\> \> \> \> \> \> \> cm$^{{\rm 3}}$ cm$^{{\rm -3}}$\\
SPA\> \> R\> specific pod area\> \> \> \> \> \> \> ha kg$^{{\rm -1}}$\\
SPAN\> \> R\> life span of leaves growing at an average temperature of 35 \degrees C\> \> \> \> \> \> \> d\\
SSA\> \> R\> specific stem area\> \> \> \> \> \> \> ha kg$^{{\rm -1}}$\\
T       \> \> R   \> actual transpiration rate                          \> \> \> \> \> \> \> cm d$^{{\rm -1}}$\\
TADW    \> \> R   \> total above$-$ground dry weight (living) of crop     \> \> \> \> \> \> \> kg ha$^{{\rm -1}}$\\
TAGP\> \> R\> total above-ground dry weight of dead and living plant organs\> \> \> \> \> \> \> kg ha$^{{\rm -1}}$\\
TAGP1\> \> R\> total above-ground dry weight of dead and living plant organs \> \> \> \> \> \> \> kg ha$^{{\rm -1}}$\\
\>\> \> (output variable)\\
TAGPX\> \> R\> total above-ground dry weight of dead and living plant organs \> \> \> \> \> \> \> kg ha$^{{\rm -1}}$\\
\>\> \> (summary variable)\\
TBASE\> \> R\> lower threshold temperature for physiological ageing of leaves\> \> \> \> \> \> \> \degrees C \\
TBASEM\> \> R\> lower threshold below which phenological development stops \> \> \> \> \> \> \> \degrees C\\
TDWI\> \> R\> initial total dry weight of crop\> \> \> \> \> \> \> kg ha$^{{\rm -1}}$ \\
TEFF    \> \> R   \> temperature effect on maintenance respiration      \> \> \> \> \> \> \> $-$\\
TEFFMX\> \> R\> maximum effective temperature for emergence\> \> \> \> \> \> \> \degrees C\\
TEMP    \> \> R   \> average daily air temperature                      \> \> \> \> \> \> \> \degrees C\\
TERMNL\> \> L\> flag to indicate to stop the simulation\> \> \> \> \> \> \> -\\
TIME\> \> R\> day number controlled by the timer routine\> \> \> \> \> \> \> -\\
TMAX    \> \> R   \> maximum daily air temperature                      \> \> \> \> \> \> \> \degrees C\\
TMIN    \> \> R   \> minimum daily air temperature                      \> \> \> \> \> \> \> \degrees C\\
TMINRA  \> \> R   \> seven days running average of minimum daily air temperature   \> \> \> \> \> \> \> \degrees C\\
TMNFTB\> \> R\> correction factor of daily gross CO$_{{\rm 2}}$ assimilation rate as a\\
\>\> \> function of low minimum temperatures (AFGEN table)\> \> \> \> \> \> \> \degrees C\\
TMNSAV\> \> R\> seven day running average of minimum temperature\> \> \> \> \> \> \> \degrees C\\
TMPFTB\> \> R\> correction factor of maximum leaf CO$_{{\rm 2}}$ assimilation rate as a\\
\>\> \> function of sub-optimum average day temperatures (AFGEN table)\> \> \> \> \> \> \> \degrees C\\
TRA\> \> R\> crop transpiration rate\> \> \> \> \> \> \> cm d$^{{\rm -1}}$\\
TRA1\> \> R\> crop transpiration rate (output variable)\> \> \> \> \> \> \> cm d$^{{\rm -1}}$\\
TRAJWL\> \> R\> subtotal water limited transpiration\\
TRAMX\> \> R\> maximum crop transpiration rate\> \> \> \> \> \> \> cm d$^{{\rm -1}}$\\
TRAT\> \> R\> total crop transpiration\> \> \> \> \> \> \> cm\\
TRATPP\> \> R\> total crop transpiration, potential production\> \> \> \> \> \> \> cm\\
TRATWL\> \> R\> total crop transpiration, water limited production\> \> \> \> \> \> \> cm\\
TRATX\> \> R\> total crop transpiration (summary variable)\> \> \> \> \> \> \> cm\\
TRC    \> \> R   \> transpiration coefficient  \> \> \> \> \> \> \> kg kg$^{{\rm -1}}$\\
TRCPP\> \> R   \> transpiration coefficient, potential production  \> \> \> \> \> \> \> kg kg$^{{\rm -1}}$\\
TRCWL\> \> R   \> transpiration coefficient, water limited production  \> \> \> \> \> \> \> kg kg$^{{\rm -1}}$\\
TRCX\> \> R   \> transpiration coefficient (output variable)\> \> \> \> \> \> \> kg kg$^{{\rm -1}}$\\
TSUM\> \> R\> temperature sum\> \> \> \> \> \> \> \degrees C\\
TSUM1\> \> R\> threshold temperature sum from emergence to anthesis\> \> \> \> \> \> \> \degrees C\\
TSUM2\> \> R\> threshold temperature sum from anthesis to maturity\> \> \> \> \> \> \> \degrees C\\
TSUME\> \> R\> temperature sum from sowing to emergence\> \> \> \> \> \> \> \degrees C\\
TSUMEM\> \> R\> threshold temperature sum from sowing to emergence\> \> \> \> \> \> \> \degrees C\\
TWLV\> \> R\> dry weight of dead and living leaves\> \> \> \> \> \> \> kg ha$^{{\rm -1}}$\\
TWLVX\> \> R\> dry weight of dead and living leaves (output variable)\> \> \> \> \> \> \> kg ha$^{{\rm -1}}$\\
TWRT\> \> R\> dry weight of dead and living roots \> \> \> \> \> \> \> kg ha$^{{\rm -1}}$\\
TWRTX\> \> R\> dry weight of dead and living roots (output variable)\> \> \> \> \> \> \> kg ha$^{{\rm -1}}$\\
TWSO\> \> R\> dry weight of dead and living storage organs\> \> \> \> \> \> \> kg ha$^{{\rm -1}}$\\
TWSOX\> \> R\> dry weight of dead and living storage organs (output variable)\> \> \> \> \> \> \> kg ha$^{{\rm -1}}$\\
TWST\> \> R\> dry weight of dead and living stems \> \> \> \> \> \> \> kg ha$^{{\rm -1}}$\\
TWSTX\> \> R\> dry weight of dead and living stems (output variable)\> \> \> \> \> \> \> kg ha$^{{\rm -1}}$\\
VAR\_NO\> \> I\> variety number\> \> \> \> \> \> \> -\\
WLV\> \> R\> dry weight of living leaves\> \> \> \> \> \> \> kg ha$^{{\rm -1}}$\\
WLV1\> \> R\> dry weight of living leaves (output variable)\> \> \> \> \> \> \> kg ha$^{{\rm -1}}$\\
WRT     \> \> R   \> dry weight of living roots                         \> \> \> \> \> \> \> kg ha$^{{\rm -1}}$\\
WSO     \> \> R   \> dry weight of living storage organs                \> \> \> \> \> \> \> kg ha$^{{\rm -1}}$\\
WSO1\> \> R\> dry weight of living storage organs (output variable)\> \> \> \> \> \> \> kg ha$^{{\rm -1}}$\\
WST\> \> R   \> dry weight of living stems              \> \> \> \> \> \> \> kg ha$^{{\rm -1}}$\\
WST1\> \> R\> dry weight of living stem (output variable)\> \> \> \> \> \> \> kg ha$^{{\rm -1}}$\\
YAGPP\> \> R\> total above-ground dry weight of dead and living\\
\>\> \> plant organs, potential production (output variable)\> \> \> \> \> \> \> kg ha$^{{\rm -1}}$\\
YAGWL\> \> R\> total above-ground dry weight of dead and living\\
\>\> \> plant organs, water limited production (output variable)\> \> \> \> \> \> \> kg ha$^{{\rm -1}}$\\
YLVPP\> \> R\> dry weight of dead and living leaves, \\
\>\> \> potential production (output variable)\> \> \> \> \> \> \> kg ha$^{{\rm -1}}$\\
YLVWL\> \> R\> dry weight of dead and living leaves, \\
\>\> \> water limited production (output variable)\> \> \> \> \> \> \> kg ha$^{{\rm -1}}$\\
YRTPP\> \> R\> dry weight of dead and living roots, \\
\>\> \> potential production (output variable)\> \> \> \> \> \> \> kg ha$^{{\rm -1}}$\\
YRTWL\> \> R\> dry weight of dead and living roots, \\
\>\> \> water limited production (output variable)\> \> \> \> \> \> \> kg ha$^{{\rm -1}}$\\
YSOPP\> \> R\> dry weight of dead and living storage organs,\\
\>\> \> potential production (output variable)\> \> \> \> \> \> \> kg ha$^{{\rm -1}}$\\
YSOWL\> \> R\> dry weight of dead and living storage organs, \\
\>\> \> water limited production (output variable)\> \> \> \> \> \> \> kg ha$^{{\rm -1}}$\\
YSTPP\> \> R\> dry weight of dead and living stems,\\
\>\> \> potential production (output variable)\> \> \> \> \> \> \> kg ha$^{{\rm -1}}$\\
YSTWL\> \> R\> dry weight of dead and living stems, \\
\>\> \> water limited production (output variable)\> \> \> \> \> \> \> kg ha$^{{\rm -1}}$
\end{tabbing}

\bigskip
\bigskip
\bigskip
\bigskip
{\bf SUBROUTINE EVTRA}
\testlastline

\begin{indenting}{3.81cm}
(IWB, IOX, IAIRDU, KDIF, CFET, DEPNR, E0, ES0, ET0, LAI, SM, SMO, SMFCF,
\end{indenting}
\begin{tabbing}
\hspace{1.27cm}\=\hspace{1.27cm}\=\hspace{1.27cm}\=\hspace{1.27cm}\=%
\hspace{1.27cm}\=\hspace{1.27cm}\=\hspace{1.27cm}\=\hspace{1.27cm}\=%
\hspace{1.27cm}\=\hspace{1.27cm}\=\kill
 \>\> \>  SMW, CRAIRC, EVWMX, EVSMX, TRAMX, TRA, IDOS, IDWS)
\end{tabbing}
\nwln
\begin{tabbing}
\hspace{1.27cm}\=\hspace{1.27cm}\=\hspace{1.27cm}\=\hspace{1.27cm}\=%
\hspace{1.27cm}\=\hspace{1.27cm}\=\hspace{1.27cm}\=\hspace{1.27cm}\=%
\hspace{1.27cm}\=\hspace{1.27cm}\=\kill
Name    \> \> Type   \> Description                                        \> \> \> \> \> \> \> Units\\
$-$$-$$-$$-$    \> \> $-$$-$$-$$-$   \> $-$$-$$-$$-$$-$$-$$-$$-$$-$$-$$-$                                        \> \> \> \> \> \> \> $-$$-$$-$$-$$-$\\
CFET\> \> R\> crop specific correction parameter of potential evapotranspiration rate\> \> \> \> \> \> \> -\\
CRAIRC\> \> R\> critical air content\> \> \> \> \> \> \> cm$^{{\rm 3}}$ cm$^{{\rm -3}}$\\
DEPNR\> \> R   \> crop group number (from 1(= drought sensitive) to \\
\>\> \> 5(= drought resistent))   \> \> \> \> \> \> \> $-$\\
DSOS\> \> R\> days since oxygen shortage\> \> \> \> \> \> \> d\\
E0\> \> R\> evaporation rate from a free water surface\> \> \> \> \> \> \> mm d$^{{\rm -1}}$\\
EKL\> \> R\> reduction factor of the evapotranspiration due to shading\> \> \> \> \> \> \> -\\
ES0\> \> R\> evaporation rate from a bare soil surface\> \> \> \> \> \> \> mm d$^{{\rm -1}}$\\
ET0\> \> R\> potential evapotranspiration rate\> \> \> \> \> \> \> mm d$^{{\rm -1}}$\\
EVSMX\> \> R\> maximum evaporation rate from a shaded wet soil surface\> \> \> \> \> \> \> mm d$^{{\rm -1}}$\\
EVWMX\> \> R\> maximum evaporation rate from a shaded water surface\> \> \> \> \> \> \> mm d$^{{\rm -1}}$\\
IAIRDU\> \> I\> flag; indicates presence (1) or absence (0) of airducts in the plant \> \> \> \> \> \> \> -\\
IDOS\> \> I\> number of days with oxygen shortage\> \> \> \> \> \> \> d\\
IDWS\> \> I\> number of days with water shortage\> \> \> \> \> \> \> d\\
IOX\> \> I\> flag; controlling calculation of potential (0) or water limited yield (1) \> \> \> \> \> \> \> -\\
IWB\> \> I\> flag; controlling calculation of water limited yield without (0) or \\
\>\> \> with (1) accounting for oxygen shortage in the root zone\> \> \> \> \> \> \> -\\
KDIF\> \> R\> extinction coefficient for diffuse radiation\> \> \> \> \> \> \> -\\
KGLOB\> \> R\> extinction coefficient for total global radiation\> \> \> \> \> \> \> -\\
LAI\> \> R\> leaf area index\> \> \> \> \> \> \> ha ha$^{{\rm -1}}$\\
RFOS\> \> R\> reduction factor for transpiration in case of oxygen shortage\> \> \> \> \> \> \> -\\
RFOSMX\> \> R\> max. reduction factor for transpiration in case of oxygen shortage\> \> \> \> \> \> \> -\\
RFWS\> \> R\> reduction factor for transpiration in case of water shortage\> \> \> \> \> \> \> -\\
SM\> \> R\> soil moisture content\> \> \> \> \> \> \> cm$^{{\rm 3}}$ cm$^{{\rm -3}}$\\
SM0\> \> R\> soil porosity, saturated moisture content \> \> \> \> \> \> \> cm$^{{\rm 3}}$ cm$^{{\rm -3}}$\\
SMAIR\> \> R\> critical soil moisture content for aeration\> \> \> \> \> \> \> cm$^{{\rm 3}}$ cm$^{{\rm -3}}$\\
SMCR\> \> R \> critical soil moisture content\> \> \> \> \> \> \> cm$^{{\rm 3}}$ cm$^{{\rm -3}}$\\
SMFCF\> \> R\> soil moisture content at field capacity\> \> \> \> \> \> \> cm$^{{\rm 3}}$ cm$^{{\rm -3}}$\\
SMW\> \> R \> soil moisture content at wilting point\> \> \> \> \> \> \> cm$^{{\rm 3}}$ cm$^{{\rm -3}}$\\
SWDEP\> \> R\> soil water depletion factor\> \> \> \> \> \> \> -\\
TRA\> \> R\> actual transpiration rate\> \> \> \> \> \> \> mm d$^{{\rm -1}}$\\
TRAMX\> \> R\> maximum transpiration rate\> \> \> \> \> \> \> mm d$^{{\rm -1}}$
\end{tabbing}

\bigskip
\bigskip
\bigskip
\bigskip
{\bf SUBROUTINE PENMAN}
\testlastline

\begin{indenting}{3.81cm}
(IDAY, LAT, ELEV, AGSTA, ANGSTB, TMIN, TMAX, AVRAD, VAP,
\end{indenting}
\begin{tabbing}
\hspace{1.27cm}\=\hspace{1.27cm}\=\hspace{1.27cm}\=\hspace{1.27cm}\=%
\hspace{1.27cm}\=\hspace{1.27cm}\=\hspace{1.27cm}\=\hspace{1.27cm}\=%
\hspace{1.27cm}\=\hspace{1.27cm}\=\kill
 \>\> \> \>  WIND2, E0, ES0, ET0)
\end{tabbing}
\nwln
\begin{tabbing}
\hspace{1.27cm}\=\hspace{1.27cm}\=\hspace{1.27cm}\=\hspace{1.27cm}\=%
\hspace{1.27cm}\=\hspace{1.27cm}\=\hspace{1.27cm}\=\hspace{1.27cm}\=%
\hspace{1.27cm}\=\hspace{1.27cm}\=\kill
Name   \> Type \> Description                                        \> \> \> \> \> \> \> \> Units\\
-$-$$-$$-$   \> \> $-$$-$$-$$-$    \> $-$$-$$-$$-$$-$$-$$-$$-$$-$$-$$-$                                        \> \> \> \> \> \> \> $-$$-$$-$$-$$-$\\
ANGSTA \> \> R   \> first empirical constant in \AA ngstr\"{o}m formula             \> \> \> \> \> \> \> $-$\\
ANGSTB \> \> R   \> second empirical constant in \AA ngstr\"{o}m formula             \> \> \> \> \> \> \> $-$\\
ATMTR   \> \> R   \> fraction of Angot's radiation actually received    \> \> \> \> \> \> \> $-$\\
AVRAD   \> \> R   \> shortwave radiation actually received              \> \> \> \> \> \> \> J m$^{{\rm -2}}$ d$^{{\rm -1}}$\\
BU      \> \> R   \> empirical constant of the wind function in the Penman formula  \> \> \> \> \> \> \> s m$^{{\rm -1}}$  \\
COSLD   \> \> R   \> cosine(latitude) times cosine(declination of the sun)       \> \> \> \> \> \> \> $-$\\
DAYL    \> \> R   \> day length                                         \> \> \> \> \> \> \> h\\
DAYLP   \> \> R   \> photoperiodic day length                           \> \> \> \> \> \> \> h\\
DELTA   \> \> R   \> slope of saturation vapor pressure curve between average air \\
\>\> \> temperature and dew point \> \> \> \> \> \> \> mbar \degrees C$^{{\rm -1}}$\\
DIFPP\> \> R\> diffuse radiation perpendicular to direction of light\> \> \> \> \> \> \> J m$^{{\rm -2}}$ s$^{{\rm -1}}$\\
DSINBE\> \> R  \> integral over the sine of solar elevation with\\
\>\> \> a correction for lower atmospheric transmission\> \> \> \> \> \> \> s\\
E0      \> \> R   \> Penman potential evaporation rate from a free water surface       \> \> \> \> \> \> \> mm d$^{{\rm -1}}$ \\
EA      \> \> R   \> evaporative demand of the atmosphere               \> \> \> \> \> \> \> mm d$^{{\rm -1}}$  \\
EAC     \> \> R   \> evaporative demand of the atmosphere above a crop canopy\> \> \> \> \> \> \> mm d$^{{\rm -1}}$  \\
ELEV    \> \> R   \> elevation above sea level                          \> \> \> \> \> \> \> m\\
ES0     \> \> R   \> Penman potential evaporation rate from a bare soil surface       \> \> \> \> \> \> \> mm d$^{{\rm -1}}$ \\
ET0     \> \> R   \> Penman potential transpiration rate from a crop canopy    \> \> \> \> \> \> \> mm d$^{{\rm -1}}$\\
GAMMA   \> \> R   \> psychrometer constant                              \> \> \> \> \> \> \> mbar \degrees C$^{{\rm -1}}$\\
IDAY    \> \> I   \> day number in the Julian calendar (between 1 and 365)    \> \> \> \> \> \> \> $-$\\
LAT     \> \> R   \> geographical latitude of location                  \> \> \> \> \> \> \> degrees\\
LHVAP   \> \> R   \> latent heat of vaporization of water               \> \> \> \> \> \> \> J kg$^{{\rm -1}}$\\
PBAR    \> \> R   \> atmospheric pressure                               \> \> \> \> \> \> \> mbar\\
PSYCON  \> \> R   \> psychrometric instrument constant                  \> \> \> \> \> \> \> \degrees C$^{{\rm -1}}$\\
RB      \> \> R   \> net outgoing longwave radiation                    \> \> \> \> \> \> \> J m$^{{\rm -2}}$ d$^{{\rm -1}}$\\
REFCFC  \> \> R   \> albedo for crop canopy                             \> \> \> \> \> \> \> $-$\\
REFCFS  \> \> R   \> albedo for bare soil surface                       \> \> \> \> \> \> \> $-$\\
REFCFW  \> \> R   \> albedo for water surface                           \> \> \> \> \> \> \> $-$\\
RELSSD  \> \> R   \> ratio of actual sunshine duration to max. duration on a cloudless day\> \> \> \> \> \> \> $-$\\
RNC     \> \> R  \> net radiation absorbed by a crop canopy            \> \> \> \> \> \> \> mm d$^{{\rm -1}}$  \\
RNS     \> \> R  \> net radiation absorbed by a bare soil surface      \> \> \> \> \> \> \> mm d$^{{\rm -1}}$  \\
RNW     \> \> R   \> net radiation absorbed by a free water surface     \> \> \> \> \> \> \> mm d$^{{\rm -1}}$\\
SINLD   \> \> R   \> sine(latitude) times sine(declination of the sun)  \> \> \> \> \> \> \> $-$\\
STBC    \> \> R   \> Stefan Boltzman constant                           \> \> \> \> \> \> \> J m$^{{\rm -1}}$ d$^{{\rm -1}}$ K$^{{\rm -4}}$\\
SVAP    \> \> R   \> saturated vapor pressure                          \> \> \> \> \> \> \> mbar\\
TDIF    \> \> R   \> difference between maximum and minimum daily air temperature\> \> \> \> \> \> \> \degrees C\\
TMAX    \> \> R   \> maximum daily air temperature                      \> \> \> \> \> \> \> \degrees C
\end{tabbing}

\nwln
\begin{tabbing}
\hspace{1.27cm}\=\hspace{1.27cm}\=\hspace{1.27cm}\=\hspace{1.27cm}\=%
\hspace{1.27cm}\=\hspace{1.27cm}\=\hspace{1.27cm}\=\hspace{1.27cm}\=%
\hspace{1.27cm}\=\hspace{1.27cm}\=\kill
TMIN    \> \> R   \> minimum daily air temperature                      \> \> \> \> \> \> \> \degrees C\\
TMPA    \> \> R   \> mean daily air temperature                         \> \> \> \> \> \> \> \degrees C\\
VAP     \> \> R   \> 24 hours average vapor pressure                      \> \> \> \> \> \> \> mbar\\
WIND2    \> \> R   \> 24 hours average wind speed at 2 meters        \> \> \> \> \> \> \> m s$^{{\rm -1}}$
\end{tabbing}

\bigskip
\bigskip
\bigskip
\nwln
\begin{tabbing}
\hspace{1.27cm}\=\hspace{1.27cm}\=\hspace{1.27cm}\=\hspace{1.27cm}\=%
\hspace{1.27cm}\=\hspace{1.27cm}\=\hspace{1.27cm}\=\hspace{1.27cm}\=%
\hspace{1.27cm}\=\hspace{1.27cm}\=\kill
{\bf SUBROUTINE ROOTD}\> \> \> (ITASK, DELT, IWB, IZT, FR, RRI, IAIRDU, RDI, RDMCR, RDMSOL, ZTI,\\
 \>\> \>  ZT, RDM, RD)
\end{tabbing}
\nwln
\begin{tabbing}
\hspace{1.27cm}\=\hspace{1.27cm}\=\hspace{1.27cm}\=\hspace{1.27cm}\=%
\hspace{1.27cm}\=\hspace{1.27cm}\=\hspace{1.27cm}\=\hspace{1.27cm}\=%
\hspace{1.27cm}\=\hspace{1.27cm}\=\kill
Name    \> \> Type   \> Description                                        \> \> \> \> \> \> \> Units\\
$-$$-$$-$$-$    \> \> $-$$-$$-$$-$   \> $-$$-$$-$$-$$-$$-$$-$$-$$-$$-$$-$                                        \> \> \> \> \> \> \> $-$$-$$-$$-$$-$\\
DELT\> \> R\> time step\\
FR      \> \> R   \> fraction of dry$-$matter increase partitioned to the roots      \> \> \> \> \> \> \> $-$\\
IAIRDU\> \> I\> flag; (0) no airducts (1) airducts exist\> \> \> \> \> \> \> -\\
ITASK\> \> I\> task which the subroutine should perform\> \> \> \> \> \> \> -\\
\>\> \> ITASK = 1, initialization\\
\>\> \> ITASK = 2, rate calculation\\
\>\> \> ITASK = 3, integration\\
\>\> \> ITASK = 4, finish section\\
ITOLD\> \> I\> retains number of last task performed\> \> \> \> \> \> \> -\\
IWB\> \> I\> flag; controlling calculation of water limited yield without (0) or \\
\>\> \> with (1) accounting for oxygen shortage in the root zone\> \> \> \> \> \> \> -\\
IZT\> \> I\> flag; (0) no ground water influence, (1) ground water influence\> \> \> \> \> \> \> -\\
RD\> \> R\> rooting depth\> \> \> \> \> \> \> cm\\
RDI\> \> R\> initial rooting depth\> \> \> \> \> \> \> cm\\
RDM\> \> R\> maximum rooting depth\> \> \> \> \> \> \> cm\\
RDMCR\> \> R\> maximum rooting depth provided by user\> \> \> \> \> \> \> cm\\
RDMO\> \> R\> maximum rooting depth reach under potential production\> \> \> \> \> \> \> cm\\
RDMSOL\> \> R\> maximum rootable soil depth\> \> \> \> \> \> \> cm\\
RR\> \> R\> root growth (not considered as a rate)\> \> \> \> \> \> \> cm \\
RRI\> \> R\> maximum daily increase of rooting depth\> \> \> \> \> \> \> cm d$^{{\rm -1}}$\\
ZT\> \> R\> ground water depth\> \> \> \> \> \> \> cm\\
ZTI\> \> R\> initial ground water depth\> \> \> \> \> \> \> cm
\end{tabbing}

\bigskip
\bigskip
\bigskip
\nwln
\begin{tabbing}
\hspace{1.27cm}\=\hspace{1.27cm}\=\hspace{1.27cm}\=\hspace{1.27cm}\=%
\hspace{1.27cm}\=\hspace{1.27cm}\=\hspace{1.27cm}\=\hspace{1.27cm}\=%
\hspace{1.27cm}\=\hspace{1.27cm}\=\kill
{\bf SUBROUTINE STDAY}\> \> \> (ITASK, CRPNAM, SOFILE, IUSO  , IUOUT , IULOG, RAIN, ES0,\\
\>\> \>  DAY , IDESOW, IDLSOW, ISTATE, COSUT)
\end{tabbing}
\nwln
\begin{tabbing}
\hspace{1.27cm}\=\hspace{1.27cm}\=\hspace{1.27cm}\=\hspace{1.27cm}\=%
\hspace{1.27cm}\=\hspace{1.27cm}\=\hspace{1.27cm}\=\hspace{1.27cm}\=%
\hspace{1.27cm}\=\hspace{1.27cm}\=\kill
Name    \> \> Type   \> Description                                        \> \> \> \> \> \> \> Units\\
-$-$$-$$-$    \> \> $-$$-$$-$$-$   \> $-$$-$$-$$-$$-$$-$$-$$-$$-$$-$$-$                                       \> \> \> \> \> \> \> $-$$-$$-$$-$$-$\\
CAPRMX     \> \> R\> maximum upward flow into plow layer\> \> \> \> \> \> \> cm \\
CAPRFU(10) \> \> R\> upward flow as a function of negative values of WEXC, when\\
\>\> \> topsoil is drier than field capacity\> \> \> \> \> \> \> cm \\
COSUT      \> \> I\> counts times that sowing date equals latest sowing date\\
\>\> \> (indicator for the suitability of a soil for a specific crop).\\
DEFLIM     \> \> R\> minimum required soil moisture deficit in plow layer\> \> \> \> \> \> \> cm\\
          \>\> \> for occurrence of workable day (workability criterion)\> \> \> \> \> \> \> cm\\
EVS\> \> R\> daily evaporation from bare soil surface\> \> \> \> \> \> \> cm\\
IDAY\> \> I\> Julian date\> \> \> \> \> \> \> -\\
IDFWOR\> \> I\> first workable day\> \> \> \> \> \> \> -\\
IDESOW\> \> I\> earliest sowing date\> \> \> \> \> \> \> -\\
IDLSOW\> \> I\> latest sowing date\> \> \> \> \> \> \> -\\
IDSOW\> \> I\> sowing date\> \> \> \> \> \> \> -\\
ILWPER\> \> I\> length of workable period, should be 3 for sowing\> \> \> \> \> \> \> -\\
\>\> \> (sowing criterion)\\
RAIN\> \> R\> daily rainfall\> \> \> \> \> \> \> cm\\
SEEP\> \> R\> daily seepage from plow layer\> \> \> \> \> \> \> cm\\
SPADS   \> \> R\> first topsoil seepage parameter for deep seedbed (potato)\> \> \> \> \> \> \> -\\
SPODS\> \> R\> second topsoil seepage parameter for deep seedbed (potato)\> \> \> \> \> \> \> -\\
SPASS\> \> R\> first topsoil seepage parameter for shallow seedbed\> \> \> \> \> \> \> -\\
SPOSS\> \> R\> second topsoil seepage parameter for shallow seedbed\> \> \> \> \> \> \> -\\
SPAC\> \> R\> first topsoil seepage parameter\> \> \> \> \> \> \> -\\
SPOC\> \> R\> second topsoil seepage parameter\> \> \> \> \> \> \> -\\
WEXC\> \> R\> excess amount of water in plow layer\> \> \> \> \> \> \> cm
\end{tabbing}

\bigskip
\bigskip
\bigskip
\nwln
\begin{tabbing}
\hspace{1.27cm}\=\hspace{1.27cm}\=\hspace{1.27cm}\=\hspace{1.27cm}\=%
\hspace{1.27cm}\=\hspace{1.27cm}\=\hspace{1.27cm}\=\hspace{1.27cm}\=%
\hspace{1.27cm}\=\hspace{1.27cm}\=\kill
{\bf SUBROUTINE SUBSOL} \> \> \> (PF, D, FLOW, CONTAB, ILCON)
\end{tabbing}
\nwln
\begin{tabbing}
\hspace{1.27cm}\=\hspace{1.27cm}\=\hspace{1.27cm}\=\hspace{1.27cm}\=%
\hspace{1.27cm}\=\hspace{1.27cm}\=\hspace{1.27cm}\=\hspace{1.27cm}\=%
\hspace{1.27cm}\=\hspace{1.27cm}\=\kill
Name    \> \> Type   \> Description                                        \> \> \> \> \> \> \> Units\\
-$-$$-$$-$    \> \> $-$$-$$-$$-$   \> $-$$-$$-$$-$$-$$-$$-$$-$$-$$-$$-$                                       \> \> \> \> \> \> \> $-$$-$$-$$-$$-$\\
CONDUC  \> \> R   \> array with conductivity values belonging to the matric head values used for \\
\>\> \> function evaluation in the integration process              \> \> \> \> \> \> \> cm d$^{{\rm -1}}$\\
CONTAB  \> \> R   \> the $^{{\rm 10}}$log of conductivity as a function of the pF (AFGEN table)    \> \> \> \> \> \> \> log(cm/d)\\
D       \> \> R   \> distance between groundwater and lower boundary of the root zone      \> \> \> \> \> \> \> cm\\
D1      \> \> R   \> local variable (no formal parameter) equal to D    \> \> \> \> \> \> \> cm\\
DEL     \> \> R   \> array containing ranges of at most four matric head integration \\
\>\> \> intervals \> \> \> \> \> \> \> cm or log(cm)\\
DF      \> \> R   \> accuracy of estimated flow during iteration        \> \> \> \> \> \> \> cm d$^{{\rm -1}}$ \\
ELOG10  \> \> R   \> mathematical constant (= ln(10))                  \> \> \> \> \> \> \> $-$\\
FL      \> \> R   \> lower bound of flow rate                           \> \> \> \> \> \> \> cm d$^{{\rm -1}}$ \\
FLOW    \> \> R   \> flow rate, equal to FLW, output variable           \> \> \> \> \> \> \> cm d$^{{\rm -1}}$ \\
FLW     \> \> R   \> estimated flow rate                                \> \> \> \> \> \> \> cm d$^{{\rm -1}}$ \\
FU      \> \> R   \> upper bound of flow rate                           \> \> \> \> \> \> \> cm d$^{{\rm -1}}$ \\
HULP    \> \> R   \> array with auxiliary variables, used to save execution time during iteration \> \> \> \> \> \> \> cm$^{{\rm 2}}$ d$^{{\rm -1}}$ \\
I1      \> \> I   \> DO$-$loop control variable                           \> \> \> \> \> \> \> -\\
I2      \> \> I   \> DO$-$loop control variable                           \> \> \> \> \> \> \> $-$\\
I3      \> \> I   \> DO$-$loop control variable                           \> \> \> \> \> \> \> $-$\\
IINT    \> \> I   \> number of integration intervals                    \> \> \> \> \> \> \> $-$\\
ILCON\> \> I\> number of elements in the table CONTAB\> \> \> \> \> \> \> -\\
IMAX    \> \> I   \> number of "points" (=3*IINT)                       \> \> \> \> \> \> \> $-$\\
K0      \> \> R   \> hydraulic conductivity of saturated soil                     \> \> \> \> \> \> \> cm d$^{{\rm -1}}$ \\
LOGST4  \> \> R  \> mathematical constant, equal to $^{{\rm 10}}$log(330)         \> \> \> \> \> \> \> log(cm)\\
MH      \> \> R   \> matric head of root zone                            \> \> \> \> \> \> \> cm\\
PF      \> \> R   \> pF value belonging to MH (i.e. $^{{\rm 10}}$log(MH))          \> \> \> \> \> \> \> log(cm)\\
PF1     \> \> R   \> equals PF, not a formal parameter                  \> \> \> \> \> \> \> log(cm)\\
PFGAU   \> \> R   \> array with pF values belonging to the matric head values used for function \\
\>\> \> evaluation in the integration process                         \> \> \> \> \> \> \> log(cm)\\
PFSTAN  \> \> R   \> values for PFGAU belonging to the standard intervals for integration\\
            \>\>\> (0, 45), (45, 170) and (170, 330)                  \> \> \> \> \> \> \> log(cm)\\
PGAU    \> \> R   \> mathematical constants used for three$-$point Gaussian integration            \> \> \> \> \> \> \> $-$\\
START   \> \> R   \> array containing standard interval bounds for Gaussian integration           \> \> \> \> \> \> \> cm\\
WGAU    \> \> R   \> mathematical constants used for three$-$point Gaussian integration           \> \> \> \> \> \> \> $-$\\
Z       \> \> R   \> distance between ground water and lower boundary of root zone\> \> \> \> \> \> \> cm
\end{tabbing}

\bigskip
\bigskip
\bigskip
\bigskip
\bigskip
\bigskip
\bigskip
\bigskip
\nwln
\begin{tabbing}
\hspace{1.27cm}\=\hspace{1.27cm}\=\hspace{1.27cm}\=\hspace{1.27cm}\=%
\hspace{1.27cm}\=\hspace{1.27cm}\=\hspace{1.27cm}\=\hspace{1.27cm}\=%
\hspace{1.27cm}\=\hspace{1.27cm}\=\kill
{\bf SUBROUTINE TOTASS}\> \> \> (DAYL, AMAX, EFF, LAI, KDIF, AVRAD, DIFPP, DSINBE, SINLD, COSLD,\\
 \>\> \>  DTGA)
\end{tabbing}
\nwln
\begin{tabbing}
\hspace{1.27cm}\=\hspace{1.27cm}\=\hspace{1.27cm}\=\hspace{1.27cm}\=%
\hspace{1.27cm}\=\hspace{1.27cm}\=\hspace{1.27cm}\=\hspace{1.27cm}\=%
\hspace{1.27cm}\=\hspace{1.27cm}\=\kill
Name    \> \> Type   \> Description                                        \> \> \> \> \> \> \> Units\\
-$-$$-$$-$    \> \> $-$$-$$-$$-$   \> $-$$-$$-$$-$$-$$-$$-$$-$$-$$-$$-$                                        \> \> \> \> \> \> \> $-$$-$$-$$-$$-$\\
AMAX    \> \> R   \> maximum leaf CO$_{{\rm 2}}$ assimilation rate (light saturation)\> \> \> \> \> \> \> kg ha$^{{\rm -1}}$ h$^{{\rm -1}}$\\
AVRAD   \> \> R   \> daily shortwave radiation actually received  \> \> \> \> \> \> \> J m$^{{\rm -2}}$ d$^{{\rm -1}}$\\
COSLD   \> \> R   \> cosine(latitude) times cosine(declination of the sun)           \> \> \> \> \> \> \> $-$\\
DAYL    \> \> R   \> astronomical day length\> \> \>  \> \> \> \> h\\
DIFPP\> \> R\> diffuse irradiation perpendicular to direction of light\> \> \> \> \> \> \> J m$^{{\rm -2}}$ s$^{{\rm -1}}$\\
DSINBE\> \> R\> integral over the sine of solar elevation with\\
\>\> \> a correction for lower atmospheric transmission\> \> \> \> \> \> \> s\\
DTGA    \> \> R   \> daily total gross CO$_{{\rm 2}}$  assimilation rate       \> \> \> \> \> \> \> kg ha$^{{\rm -1}}$ d$^{{\rm -1}}$ \\
EFF     \> \> R   \> initial light$-$use efficiency of CO$_{{\rm 2}}$ assimilation of single leaves \`kg ha$^{{\rm -1}}$ h$^{{\rm -1}}$ J$^{{\rm -1}}$ m$^{{\rm 2}}$ s\\
FGROS   \> \> R   \> gross CO$_{{\rm 2}}$ assimilation rate per hour of whole crop        \> \> \> \> \> \> \> kg ha$^{{\rm -1}}$ d$^{{\rm -1}}$ \\
HOUR    \> \> R   \> hour of the day                                    \> \> \> \> \> \> \> $-$\\
I1     \> \> I   \> DO$-$loop control variable                           \> \> \> \> \> \> \> $-$\\
KDIF    \> \> R   \> extinction coefficient for diffuse visible light   \> \> \> \> \> \> \> $-$\\
LAI     \> \> R   \> leaf area index                                    \> \> \> \> \> \> \> ha ha$^{{\rm -1}}$\\
PAR\> \> R\> photosynthetically active radiation\> \> \> \> \> \> \> J m$^{{\rm -2}}$ s$^{{\rm -1}}$\\
PARDIF  \> \> R   \> flux of diffuse photosynthetically active radiation       \> \> \> \> \> \> \> J m$^{{\rm -2}}$ s$^{{\rm -1}}$  \\
PARDIR  \> \> R   \> flux of direct photosynthetically active radiation  \> \> \> \> \> \> \> J m$^{{\rm -2}}$ s$^{{\rm -1}}$ \\
SINB    \> \> R   \> sine of solar elevation                            \> \> \> \> \> \> \> $-$\\
SINLD   \> \> R   \> sine(latitude) times sine(declination of the sun)  \> \> \> \> \> \> \> $-$\\
WGAUSS\> \> R\> mathematical constants used for three-point Gaussian integration\> \> \> \> \> \> \> -\\
XGAUSS\> \> R\> mathematical constants used for three-point Gaussian integration\> \> \> \> \> \> \> -
\end{tabbing}

\bigskip
\bigskip
\nwln
\begin{tabbing}
\hspace{1.27cm}\=\hspace{1.27cm}\=\hspace{1.27cm}\=\hspace{1.27cm}\=%
\hspace{1.27cm}\=\hspace{1.27cm}\=\hspace{1.27cm}\=\hspace{1.27cm}\=%
\hspace{1.27cm}\=\hspace{1.27cm}\=\kill
{\bf SUBROUTINE WATFD}\> \> \> (ITASK, DELT, IDEM, IDHALT, SPG\_NO, RDM, RD,\\
(JRC version) \> \> \>  IAIRDU, IFUNRN, SSI, SSMAX, WAV, NOTINF, EVWMX, EVSMX, TRA,\\
 \>\> \>  SMW, CRAIRC, SM, RAIN, SMO, SMFCF, CGM\_ABORT)
\end{tabbing}
\nwln
\begin{tabbing}
\hspace{1.27cm}\=\hspace{1.27cm}\=\hspace{1.27cm}\=\hspace{1.27cm}\=%
\hspace{1.27cm}\=\hspace{1.27cm}\=\hspace{1.27cm}\=\hspace{1.27cm}\=%
\hspace{1.27cm}\=\hspace{1.27cm}\=\kill
{\bf SUBROUTINE WATFD}\> \> \> (ITASK, DELT, IDEM, IDHALT, SOFILE, IUSO, IUOUT, IULOG, RDM, RD,\\
(General version) \> \> \> IAIRDU, IFUNRN, SSI, SSMAX, WAV, NOTINF, EVWMX, EVSMX, TRA,\\
 \>\> \>  SMW, CRAIRC, SM, RAIN, SMO, SMFCF)
\end{tabbing}
\nwln
\begin{tabbing}
\hspace{1.27cm}\=\hspace{1.27cm}\=\hspace{1.27cm}\=\hspace{1.27cm}\=%
\hspace{1.27cm}\=\hspace{1.27cm}\=\hspace{1.27cm}\=\hspace{1.27cm}\=%
\hspace{1.27cm}\=\hspace{1.27cm}\=\kill
Name    \> \> Type   \> Description                                        \> \> \> \> \> \> \> Units\\
$-$$-$$-$$-$    \> \> $-$$-$$-$$-$   \> $-$$-$$-$$-$$-$$-$$-$$-$$-$$-$$-$                                        \> \> \> \> \> \> \> $-$$-$$-$$-$$-$\\
AVAIL\> \> R\> amount of water available for infiltration\> \> \> \> \> \> \> cm\\
CGM\_ABORT\> \> L\> error message on database handling\> \> \> \> \> \> \> -\\
CRAIRC\> \> R\> critical air content\> \> \> \> \> \> \> cm$^{{\rm 3}}$ cm$^{{\rm -3}}$\\
CRJ\> \> R\> subtotal of actual rate of capillary rise over print interval\> \> \> \> \> \> \> cm\\
\>\> \> (not used in WATFD, value is set to 0.0)\\
DELT\> \> R\> time step = 1 day\> \> \> \> \> \> \> d\\
DMAXJ\> \> R\> subtotal of drainage rate over print interval\> \> \> \> \> \> \> cm\\
\>\> \> (not used in WATFD, value is set to 0.0)\\
DSLR\> \> R\> days since last rainfall\> \> \> \> \> \> \> -\\
DW\> \> R\> rate of change in amount of soil moisture in rooted zone\> \> \> \> \> \> \> cm d$^{{\rm -1}}$\\
DWJ\> \> R\> subtotal of variable DW over print interval\> \> \> \> \> \> \> cm  \\
DWLOW\> \> R\> rate of change in amount of soil moisture between rooted zone \\
\>\> \> and maximum rooting depth\> \> \> \> \> \> \> cm d$^{{\rm -1}}$\\
DWLOWJ\> \> R\> subtotal of variable DWLOW over print interval\> \> \> \> \> \> \> cm \\
DZJ\> \> R\> subtotal of rate of change in groundwater depth over print interval\> \> \> \> \> \> \> cm\\
EVS\> \> R\> evaporation rate from a shaded soil surface\> \> \> \> \> \> \> cm d$^{{\rm -1}}$\\
EVS1\> \> R\> daily output soil water balance variable EVS\> \> \> \> \> \> \> cm d$^{{\rm -1}}$\\
EVSJ\> \> R\> subtotal of variable EVS over print interval\> \> \> \> \> \> \> cm \\
EVSMX\> \> R\> maximum evaporation rate from a shaded soil surface\> \> \> \> \> \> \> cm d$^{{\rm -1}}$\\
EVST\> \> R\> total evaporation from a shaded soil surface\> \> \> \> \> \> \> cm\\
EVSTX\> \> R\> output cumulative water balance variable EVST\> \> \> \> \> \> \> cm\\
EVW\> \> R\> evaporation rate from a shaded water surface\> \> \> \> \> \> \> cm d$^{{\rm -1}}$\\
EVW1\> \> R\> daily output soil water balance variable EWS\> \> \> \> \> \> \> cm d$^{{\rm -1}}$\\
EVWJ\> \> R\> subtotal of variable EVW over print interval\> \> \> \> \> \> \> cm \\
EVWMX\> \> R\> maximum evaporation rate from a shaded water surface\> \> \> \> \> \> \> cm d$^{{\rm -1}}$\\
EVWT\> \> R\> total evaporation from a shaded water surface\> \> \> \> \> \> \> cm\\
EVWTX\> \> R\> output cumulative water balance variable EVWT\> \> \> \> \> \> \> cm\\
IAIRDU\> \> I\> flag for airducts in the plant; (0) absent, (1) present\> \> \> \> \> \> \> -\\
IDEM\> \> I\> day of emergence (date 1-366)\> \> \> \> \> \> \> -\\
IDHALT\> \> I\> day that simulation is halted (date 1-366)\> \> \> \> \> \> \> -\\
IFUNRN\> \> I\> flag for the calculation of non-infiltrating fraction of rainfall; (0) fixed\\
 \>\> \> at NOTINF, (1) depends on daily rainfall (via NINFTB) and NOTINF\> \> \> \> \> \> \> -\\
ITASK\> \> I\> flag to control task to be performed\> \> \> \> \> \> \> -\\
ITOLD\> \> I\> retains number of last task performed\> \> \> \> \> \> \> -\\
IULOG\> \> I\> unit number, log file\> \> \> \> \> \> \> -\\
IUOUT\> \> I\> unit number, output file\> \> \> \> \> \> \> -\\
IUSO\> \> I\> unit number, soil data file \> \> \> \> \> \> \> -\\
K0\> \> R\> conductivity of saturated soil\> \> \> \> \> \> \> cm d$^{{\rm -1}}$\\
KSUB\> \> R\> hydraulic conductivity used in calculations without groundwater\> \> \> \> \> \> \> cm d$^{{\rm -1}}$\\
LOSS\> \> R\> loss of water at lower end of the maximum root zone\> \> \> \> \> \> \> cm d$^{{\rm -1}}$\\
LOSSJ\> \> R\> subtotal of variable LOSS over print interval\> \> \> \> \> \> \> cm\\
LOSST\> \> R\> total loss of water by deep leaching\> \> \> \> \> \> \> cm\\
LOSSTX\> \> R\> output cumulative water balance variable LOSST\> \> \> \> \> \> \> cm\\
MWC\> \> R\> mean water content root zone during crop growth, end of simulation\> \> \> \> \> \> \> cm$^{{\rm 3}}$ cm$^{{\rm -3}}$\\
MWCX\> \> R\> output for cumulative water balance of variable MWC\> \> \> \> \> \> \> cm$^{{\rm 3}}$ cm$^{{\rm -3}}$\\
NINFTB\> \> R\> multiplier for non-infiltrating fraction of rainfall as function of\\
\>\> \> daily rainfall (AFGEN table)\> \> \> \> \> \> \> cm d$^{{\rm -1}}$\\
NOTINF\> \> R\> (maximum) non infiltrating fraction of rainfall\> \> \> \> \> \> \> -\\
PERC\> \> R\> actual percolation rate\> \> \> \> \> \> \> cm d$^{{\rm -1}}$\\
PERC1\> \> R\> perc. from rooted zone to subsoil (= excess of moisture root zone)\> \> \> \> \> \> \> cm d$^{{\rm -1}}$\\
PERC2\> \> R\> maximum percolation (determined by uptake capacity subsoil)\> \> \> \> \> \> \> cm d$^{{\rm -1}}$\\
PERCJ\> \> R\> subtotal of variable PERC over print interval\> \> \> \> \> \> \> cm\\
PERCT\> \> R\> total percolation\> \> \> \> \> \> \> cm\\
PERCTX\> \> R\> output cumulative water balance variable PERCT\> \> \> \> \> \> \> cm\\
RAIN\> \> R\> daily rainfall\> \> \> \> \> \> \> cm d$^{{\rm -1}}$\\
RAINJ\> \> R\> subtotal of variable RAIN over print interval\> \> \> \> \> \> \> cm\\
RAINT\> \> R\> total rainfall\> \> \> \> \> \> \> cm\\
RAINTX\> \> R\> output cumulative water balance variable RAINT\> \> \> \> \> \> \> cm\\
RD\> \> R\> rooting depth\> \> \> \> \> \> \> cm\\
RDM\> \> R\> maximum rooting depth\> \> \> \> \> \> \> cm\\
RDOLD\> \> R\> old rooting depth\> \> \> \> \> \> \> cm\\
 RELSM\> \> \> R\> subtotal relative soil moisture status\> \> \> \> \> \> -\\
RIN\> \> R\> actual infiltration rate\> \> \> \> \> \> \> cm d$^{{\rm -1}}$\\
RINJ\> \> R\> subtotal of variable RIN over print interval\> \> \> \> \> \> \> cm\\
RIRR\> \> R\> actual irrigation rate\> \> \> \> \> \> \> cm d$^{{\rm -1}}$\\
RIRRJ\> \> R\> subtotal of variable RIRR over print interval\> \> \> \> \> \> \> cm\\
SM\> \> R\> actual soil moisture content\> \> \> \> \> \> \> cm$^{{\rm 3}}$ cm$^{{\rm -3}}$\\
SM0\> \> R\> saturated moisture content; soil porosity\> \> \> \> \> \> \> cm$^{{\rm 3}}$ cm$^{{\rm -3}}$\\
SM1\> \> R\> daily output soil water balance variable SM\> \> \> \> \> \> \> cm$^{{\rm 3}}$ cm$^{{\rm -3}}$\\
SMFCF\> \> R\> soil moisture content at field capacity\> \> \> \> \> \> \> cm$^{{\rm 3}}$ cm$^{{\rm -3}}$\\
SMLIM\> \> R\> limit for initial soil moisture content\> \> \> \> \> \> \> cm$^{{\rm 3}}$ cm$^{{\rm -3}}$\\
SMW\> \> R\> soil moisture content at wilting point\> \> \> \> \> \> \> cm$^{{\rm 3}}$ cm$^{{\rm -3}}$\\
SPG\_NO\> \> I\> soil group number\> \> \> \> \> \> \> -\\
SOFILE\> \> C\> name of soil data file\> \> \> \> \> \> \> -\\
SOPE\> \> R\> conductivity of wet soil, limiting the infiltration rate in case of \\
\>\> \> surface storage, and in calculations without groundwater \> \> \> \> \> \> \> cm d$^{{\rm -1}}$\\
SS\> \> R\> actual surface storage\> \> \> \> \> \> \> cm\\
SS1\> \> R\> daily output soil water balance variable SS\> \> \> \> \> \> \> cm\\
SSI\> \> R\> initial surface storage\> \> \> \> \> \> \> cm\\
SSIX\> \> R\> output for cumulative water balance of variable SSI\> \> \> \> \> \> \> cm\\
SSX\> \> R\> output for cumulative water balance of variable SS\> \> \> \> \> \> \> cm\\
SSMAX\> \> R\> maximum surface storage\> \> \> \> \> \> \> cm\\
SSPRE\> \> R \> preliminary calculated surface storage\> \> \> \> \> \> \> cm\\
SUBTOT\> \> -\> COMMON BLOCK; subtotals summary water bal. (0 after printing)\> \> \> \> \> \> \> -\\
SUMSM\> \> R\> total soil moisture content over growing period \> \> \> \> \> \> \> cm$^{{\rm 3}}$ cm$^{{\rm -3}}$\\
TINFX\> \> R\> output cumulative water balance variable TOTINF\> \> \> \> \> \> \> cm\\
TIRRX\> \> R\> output cumulative water balance variable TOTIRR\> \> \> \> \> \> \> cm\\
TOTINF\> \> R\> total infiltration\> \> \> \> \> \> \> cm\\
TOTIRR\> \> R\> total irrigation\> \> \> \> \> \> \> cm\\
TRA\> \> R\> actual transpiration rate (as calculated in EVTRA)\> \> \> \> \> \> \> cm d$^{{\rm -1}}$\\
TRAJ\> \> R\> subtotal of variable TRA over print interval\> \> \> \> \> \> \> cm\\
TRAJWP\> \> R\> subtotal of variable TRA for water limited production\> \> \> \> \> \> \> cm\\
TRAT\> \> R\> total transpiration\> \> \> \> \> \> \> cm\\
TRATX\> \> R\> output cumulative water balance variable TRAT\> \> \> \> \> \> \> cm\\
TSR\> \> R\> total surface runoff\> \> \> \> \> \> \> cm\\
TSRX\> \> R\> output cumulative water balance variable TSR\> \> \> \> \> \> \> cm\\
TWE\> \> R\> total water content of potentially rooted zone at end of simulation\> \> \> \> \> \> \> cm$^{{\rm 3}}$ cm$^{{\rm -3}}$\\
TWEX\> \> R\> output for cumulative water balance of variable TWE\> \> \> \> \> \> \> cm$^{{\rm 3}}$ cm$^{{\rm -3}}$\\
W\> \> R\> amount of water in the rooted zone\> \> \> \> \> \> \> cm\\
WAV\> \> R\> amount of water in excess of wilting point at emergence \> \> \> \> \> \> \> cm\\
WBALFD\> \> -\> COMMON BLOCK; output variables for summary water balance\> \> \> \> \> \> \> -\\
WBALRT\> \> R\> checksum water balance for system without ground water influence\> \> \> \> \> \> \> cm\\
WBALTT\> \> R\> checksum water balance for system without ground water influence\> \> \> \> \> \> \> cm\\
WBTOTX\> \> R\> output for cumulative water balance of variable WBALTT\> \> \> \> \> \> \> cm\\
WBRTX\> \> R\> output for cumulative water balance of variable WBALRT\> \> \> \> \> \> \> cm\\
WDR\> \> R\> water added to the root zone by root growth\> \> \> \> \> \> \> cm\\
WDRT\> \> R\> total water addition to root zone by root growth\> \> \> \> \> \> \> cm\\
WDRTX\> \> R\> output cumulative water balance variable WDRT\> \> \> \> \> \> \> cm\\
WE\> \> R\> equilibrium amount of soil moisture in the rooted zone\> \> \> \> \> \> \> cm\\
WELOW\> \> R\> equilibrium amount of soil moisture below rooted zone\> \> \> \> \> \> \> cm\\
WFDDO\> \> -\> COMMON BLOCK; daily output of water variables from WATFD\> \> \> \> \> \> \> -\\
WI\> \> R\> initial amount of water in the rooted zone\> \> \> \> \> \> \> cm\\
WIX\> \> R\> output for cumulative water balance of variable WI\> \> \> \> \> \> \> cm\\
WLOW\> \> R\> amount of water between rooted zone and max. rooting depth\> \> \> \> \> \> \> cm\\
WLOWI\> \> R\> initial amount of water between rooted zone and max. root depth\> \> \> \> \> \> \> cm\\
WLOWX\> \> R\> output for cumulative water balance of variable WLOW\> \> \> \> \> \> \> cm\\
WWLOW\> \> R\> amount of water in the whole rootable zone\> \> \> \> \> \> \> cm\\
WWLOW1\> \> R\> daily output soil water balance variable WWLOW\> \> \> \> \> \> \> cm\\
WWLOWJ\> \> R\> subtotal of variable WWLOW over print interval\> \> \> \> \> \> \> cm\\
WX\> \> R\> output cumulative water balance variable W\> \> \> \> \> \> \> cm
\end{tabbing}

\bigskip
\bigskip
\bigskip
\bigskip
\nwln
\begin{tabbing}
\hspace{1.27cm}\=\hspace{1.27cm}\=\hspace{1.27cm}\=\hspace{1.27cm}\=%
\hspace{1.27cm}\=\hspace{1.27cm}\=\hspace{1.27cm}\=\hspace{1.27cm}\=%
\hspace{1.27cm}\=\hspace{1.27cm}\=\kill
{\bf SUBROUTINE WATGW}\> \> \> (ITASK, DELT, IDEM, IDHALT, TERMNL, SOFILE, IUSO, IUOUT, IULOG,\\
 \>\> \> IDRAIN, RD, IAIRDU, IFUNRN, SSI, SSMAX, ZTI, DD, NOTINF, EVWMX,\\
 \>\> \>  EVSMX, TRA, SMW, CRAIRC, ZT, SM, RAIN, SMO, SMFCF)
\end{tabbing}
\nwln
\begin{tabbing}
\hspace{1.27cm}\=\hspace{1.27cm}\=\hspace{1.27cm}\=\hspace{1.27cm}\=%
\hspace{1.27cm}\=\hspace{1.27cm}\=\hspace{1.27cm}\=\hspace{1.27cm}\=%
\hspace{1.27cm}\=\hspace{1.27cm}\=\kill
Name    \> \> Type   \> Description                                        \> \> \> \> \> \> \> Units\\
$-$$-$$-$$-$    \> \> $-$$-$$-$$-$   \> $-$$-$$-$$-$$-$$-$$-$$-$$-$$-$$-$                                        \> \> \> \> \> \> \> $-$$-$$-$$-$$-$\\
AIRC\> \> R\> air content above groundwater (groundwater in root zone)\> \> \> \> \> \> \> cm$^{{\rm 3}}$ cm$^{{\rm -3}}$\\
AVAIL\> \> R\> amount of water available for infiltration\> \> \> \> \> \> \> cm\\
CONTAB\> \> R\> table with $^{{\rm 10}}$log soil hydraulic conductivity as function of pF \> \> \> \> \> \> \> -\\
CR\> \> R\> actual rate of capillary rise\> \> \> \> \> \> \> cm d$^{{\rm -1}}$\\
CRAIRC\> \> R\> critical air content\> \> \> \> \> \> \> cm$^{{\rm 3}}$ cm$^{{\rm -3}}$\\
CRT\> \> R\> total capillary rise\> \> \> \> \> \> \> cm\\
CRTX\> \> R\> output of total capillary rise CRT\> \> \> \> \> \> \> cm\\
DD\> \> R\> distance from soil surface to drains\> \> \> \> \> \> \> cm\\
DEF1\> \> R\> new air amount between rooted zone and groundwater\> \> \> \> \> \> \> cm\\
DEFDTB\> \> R\> table with height above groundwater as function of cum. amount of air\> \> \> \> \> \> \> -\\
DELSS\> \> R\> change of surface storage from start\> \> \> \> \> \> \> cm\\
DELT\> \> R\> time step = 1 day\> \> \> \> \> \> \> d\\
DELW\> \> R\> change of the amount of water in the rooted zone from start\> \> \> \> \> \> \> cm\\
DELWZ\> \> R\> change of the amount of moisture in the soil below the rooted \\
\>\> \> zone from start\> \> \> \> \> \> \> cm\\
DMAX\> \> R\> drainage rate\> \> \> \> \> \> \> cm d$^{{\rm -1}}$\\
DR1\> \> R\> capacity of artificial drainage system\> \> \> \> \> \> \> cm d$^{{\rm -1}}$\\
DR2\> \> R\> drainage rate limited by drainable amount of water\> \> \> \> \> \> \> cm d$^{{\rm -1}}$\\
DRAINT\> \> R\> total drainage\> \> \> \> \> \> \> cm\\
DRAITX\> \> R\> output of cumulative drainage DRAINT\> \> \> \> \> \> \> cm\\
DSLR\> \> R\> days since last rainfall\> \> \> \> \> \> \> -\\
DW\> \> R\> rate of change in amount of soil moisture in rooted zone\> \> \> \> \> \> \> cm d$^{{\rm -1}}$\\
DZ\> \> R\> rate of change in groundwater depth\> \> \> \> \> \> \> cm d$^{{\rm -1}}$\\
EVS\> \> R\> evaporation rate from a shaded soil surface\> \> \> \> \> \> \> cm d$^{{\rm -1}}$\\
EVS1\> \> R\> output daily soil water balance variable EVS\> \> \> \> \> \> \> cm d$^{{\rm -1}}$\\
EVSMX\> \> R\> maximum evaporation rate from a shaded soil surface\> \> \> \> \> \> \> cm d$^{{\rm -1}}$\\
EVST\> \> R\> total evaporation from a shaded soil surface\> \> \> \> \> \> \> cm\\
EVSTX\> \> R\> output cumulative water balance variable EVST\> \> \> \> \> \> \> cm\\
EVW\> \> R\> evaporation rate from a shaded water surface\> \> \> \> \> \> \> cm d$^{{\rm -1}}$\\
EVW1\> \> R\> output of daily soil water balance variable EWS\> \> \> \> \> \> \> cm d$^{{\rm -1}}$\\
EVWMX\> \> R\> maximum evaporation rate from a shaded water surface\> \> \> \> \> \> \> cm d$^{{\rm -1}}$\\
EVWT\> \> R\> total evaporation from a shaded water surface\> \> \> \> \> \> \> cm\\
EVWTX\> \> R\> output cumulative water balance variable EVWT\> \> \> \> \> \> \> cm\\
FLOW\> \> R\> flow of water through root zone boundary\> \> \> \> \> \> \> cm d$^{{\rm -1}}$\\
I\> \> I\> DO-loop control variable\> \> \> \> \> \> \> -\\
I2\> \> I\> DO-loop control variable\> \> \> \> \> \> \> - \\
I3\> \> I\> DO-loop control variable\> \> \> \> \> \> \> - \\
I20\> \> I\> DO-loop control variable\> \> \> \> \> \> \> - \\
IAIRDU\> \> I\> flag for airducts in the plant; (0) absent, (1) present\> \> \> \> \> \> \> -\\
IDEM\> \> I\> day of emergence (Julian date, 1-366)\> \> \> \> \> \> \> -\\
IDHALT\> \> I\> day that simulation is halted (Julian date, 1-366)\> \> \> \> \> \> \> -\\
IDRAIN\> \> I\> indicates present (1) or absence of (0) of drains\> \> \> \> \> \> \> -\\
IFUNRN\> \> I\> flag for calculation of non-infiltrating fraction of rainfall; (0) fixed\\
 \>\> \> at NOTINF, (1) depends on daily rainfall (via NINFTB) and NOTINF\> \> \> \> \> \> \> -\\
ILCON\> \> I\> number of elements in table CONTAB\> \> \> \> \> \> \> -\\
ILSM\> \> I\> number of elements in table SMTAB\> \> \> \> \> \> \> -\\
ITASK\> \> I\> flag to control task to be performed\> \> \> \> \> \> \> -\\
ITOLD\> \> I\> retains number of last task performed\> \> \> \> \> \> \> -\\
IULOG\> \> I\> unit number, log file\> \> \> \> \> \> \> -\\
IUOUT\> \> I\> unit number, output file\> \> \> \> \> \> \> -\\
IUSO\> \> I\> unit number, soil data file \> \> \> \> \> \> \> -\\
K0\> \> R\> hydraulic conductivity of saturated soil\> \> \> \> \> \> \> cm d$^{{\rm -1}}$\\
MH0\> \> R\> lower matric height of integration interval\> \> \> \> \> \> \> cm\\
MH1\> \> R\> upper matric height of integration interval\> \> \> \> \> \> \> cm\\
NINFTB\> \> R\> multiplier for non-infiltrating fraction of rainfall as function of\\
\>\> \> daily rainfall (AFGEN table)\> \> \> \> \> \> \> cm d$^{{\rm -1}}$\\
NOTINF\> \> R\> (maximum) non infiltrating fraction of rainfall\> \> \> \> \> \> \> -\\
PERC\> \> R\> actual percolation rate\> \> \> \> \> \> \> cm d$^{{\rm -1}}$\\
PERCT\> \> R\> total percolation\> \> \> \> \> \> \> cm\\
PERCTX\> \> R\> output cumulative water balance variable PERCT\> \> \> \> \> \> \> cm\\
PF\> \> R\> $^{{\rm 10}}$log of hydraulic head\> \> \> \> \> \> \> -\\
PFTAB\> \> R\> table with pF data as a function of soil moisture\> \> \> \> \> \> \> -\\
PGAU\> \> R\> mathematical constant used for three point Gaussian integration\> \> \> \> \> \> \> -\\
RAIN\> \> R\> daily rainfall\> \> \> \> \> \> \> cm d$^{{\rm -1}}$\\
RAINT\> \> R\> total rainfall\> \> \> \> \> \> \> cm\\
RAINT1\> \> R\> output of daily soil water balance variable RAINT\> \> \> \> \> \> \> cm\\
RAINTX\> \> R\> output cumulative water balance variable RAINT\> \> \> \> \> \> \> cm\\
RD\> \> R\> rooting depth\> \> \> \> \> \> \> cm\\
RIN\> \> R\> actual infiltration rate\> \> \> \> \> \> \> cm d$^{{\rm -1}}$\\
RINPRE\> \> R\> preliminary infiltration rate (first estimation)\> \> \> \> \> \> \> cm d$^{{\rm -1}}$\\
RIRR\> \> R\> actual irrigation rate\> \> \> \> \> \> \> cm d$^{{\rm -1}}$\\
RTDF\> \> R\> counter of consecutive days with groundwater table within 10 cm\> \> \> \> \> \> \> -\\
SDEFTB\> \> R\> cum. amount of air as a function of height above ground water\> \> \> \> \> \> \> cm \\
SM\> \> R\> actual soil moisture content\> \> \> \> \> \> \> cm$^{{\rm 3}}$ cm$^{{\rm -3}}$\\
SM0\> \> R\> saturated moisture content; soil porosity\> \> \> \> \> \> \> cm$^{{\rm 3}}$ cm$^{{\rm -3}}$\\
SM1\> \> R\> output of daily soil water balance variable SM\> \> \> \> \> \> \> cm$^{{\rm 3}}$ cm$^{{\rm -3}}$\\
SMFCF\> \> R\> soil moisture content at field capacity\> \> \> \> \> \> \> cm$^{{\rm 3}}$ cm$^{{\rm -3}}$\\
SMTAB\> \> R\> table with soil moisture data as a function of pF (AFGEN table)\> \> \> \> \> \> \> -\\
SMW\> \> R\> soil moisture content at wilting point\> \> \> \> \> \> \> cm$^{{\rm 3}}$ cm$^{{\rm -3}}$\\
SOFILE\> \> C\> name of soil data file\> \> \> \> \> \> \> -\\
SOPE\> \> R\> conductivity of wet soil, limiting the infiltration rate in case of \\
\>\> \> surface storage \> \> \> \> \> \> \> cm d$^{{\rm -1}}$\\
SS\> \> R\> actual surface storage\> \> \> \> \> \> \> cm\\
SS1\> \> R\> output of daily soil water balance variable SS\> \> \> \> \> \> \> cm\\
SSI\> \> R\> initial surface storage\> \> \> \> \> \> \> cm\\
SSMAX\> \> R\> maximum surface storage\> \> \> \> \> \> \> cm\\
SUBAI0\> \> R\> old value of SUBAIR\\
SUBAIR\> \> R\> amount of air in soil below rooted zone\> \> \> \> \> \> \> cm\\
SUMSM\> \> R\> total soil moisture content over growing period \> \> \> \> \> \> \> cm$^{{\rm 3}}$ cm$^{{\rm -3}}$\\
TERMNL\> \> L\> logical flag\> \> \> \> \> \> \> -\\
TINFX\> \> R\> output cumulative water balance variable TOTINF\> \> \> \> \> \> \> cm\\
TIRRX\> \> R\> output cumulative water balance variable TOTIRR\> \> \> \> \> \> \> cm\\
TOTINF\> \> R\> total infiltration\> \> \> \> \> \> \> cm\\
TOTIRR\> \> R\> total irrigation\> \> \> \> \> \> \> cm\\
TRA\> \> R\> actual transpiration rate (as calculated in EVTRA)\> \> \> \> \> \> \> cm d$^{{\rm -1}}$\\
TRAT\> \> R\> total transpiration\> \> \> \> \> \> \> cm\\
TRATX\> \> R\> output cumulative water balance variable TRAT\> \> \> \> \> \> \> cm\\
TSR\> \> R\> total surface runoff\> \> \> \> \> \> \> cm\\
TSRX\> \> R\> output cumulative water balance variable TSR\> \> \> \> \> \> \> cm\\
W\> \> R\> amount of water in the rooted zone\> \> \> \> \> \> \> cm\\
WBALGW\> \> -\> COMMON BLOCK; output variables for summary water balance\> \> \> \> \> \> \> -\\
WBALRT\> \> R\> checksum water balance for system with ground water influence\> \> \> \> \> \> \> cm\\
WBALTT\> \> R\> checksum water balance for system with ground water influence\> \> \> \> \> \> \> cm\\
WBTOTX\> \> R\> output for cumulative water balance of variable WBALTT\> \> \> \> \> \> \> cm\\
WBRTX\> \> R\> output for cumulative water balance of variable WBALRT\> \> \> \> \> \> \> cm\\
WDR\> \> R\> water added to the root zone by root growth\> \> \> \> \> \> \> cm\\
WDRT\> \> R\> total water addition to root zone by root growth\> \> \> \> \> \> \> cm\\
WDRTX\> \> R\> output cumulative water balance variable WDRT\> \> \> \> \> \> \> cm\\
WE\> \> R\> equilibrium amount of soil moisture in the rooted zone\> \> \> \> \> \> \> cm\\
WEDTOT\> \> R\> equilibrium amount of moisture above drains to surface\> \> \> \> \> \> \> cm\\
WGAU\> \> R\> mathematical constant used for three point Gaussian integration\> \> \> \> \> \> \> -\\
WGWDO\> \> -\> COMMON BLOCK; daily output of water variables from WATGW\> \> \> \> \> \> \> -\\
WI\> \> R\> initial amount of water in the rooted zone\> \> \> \> \> \> \> cm\\
WIX\> \> R\> output for cumulative water balance of variable WI\> \> \> \> \> \> \> cm\\
WX\> \> R\> output cumulative water balance variable W\> \> \> \> \> \> \> cm\\
WZ\> \> R\> amount of moisture in soil below rooted zone\> \> \> \> \> \> \> cm\\
WZI\> \> R\> initial amount of moisture in soil below rooted zone\> \> \> \> \> \> \> cm\\
WZIX\> \> R\> initial amount of moisture in soil below rooted zone (output variable)\> \> \> \> \> \> \> cm\\
WZX\> \> R\> amount of moisture in soil below rooted zone (output variable)\> \> \> \> \> \> \> cm\\
XDEF\> \> R\> maximum depth of ground water table\> \> \> \> \> \> \> cm\\
ZT\> \> R\> actual depth of the groundwater table below soil surface\> \> \> \> \> \> \> cm\\
ZT1\> \> R\> output of daily soil water balance variable ZT\> \> \> \> \> \> \> cm\\
ZTI\> \> R\> initial depth of groundwater table\> \> \> \> \> \> \> cm\\
ZTMRD\> \> R\> indicator for ground water table within (-) or below (+) root zone\> \> \> \> \> \> \> -
\end{tabbing}

\bigskip
\bigskip
\bigskip
\bigskip
\bigskip
\bigskip
\bigskip
\nwln
\begin{tabbing}
\hspace{1.27cm}\=\hspace{1.27cm}\=\hspace{1.27cm}\=\hspace{1.27cm}\=%
\hspace{1.27cm}\=\hspace{1.27cm}\=\hspace{1.27cm}\=\hspace{1.27cm}\=%
\hspace{1.27cm}\=\hspace{1.27cm}\=\kill
 {\bf SUBROUTINE WATPP}\> \> \> (ITASK, DELT, SPG\_NO, IAIRDU, SM0, SMFCF,\\
(JRC version)\> \> \>  SMW, EVWMX, EVSMX, TRA, SM, CGM\_ABORT)
\end{tabbing}
\nwln
\begin{tabbing}
\hspace{1.27cm}\=\hspace{1.27cm}\=\hspace{1.27cm}\=\hspace{1.27cm}\=%
\hspace{1.27cm}\=\hspace{1.27cm}\=\hspace{1.27cm}\=\hspace{1.27cm}\=%
\hspace{1.27cm}\=\hspace{1.27cm}\=\kill
{\bf SUBROUTINE WATPP}\> \> \> (ITASK, DELT, SOFILE, IUSO, IUOUT, IULOG, IAIRDU, SM0, SMFCF,\\
(General version)\> \> \>  SMW, EVWMX, EVSMX, TRA, SM)
\end{tabbing}
\nwln
\begin{tabbing}
\hspace{1.27cm}\=\hspace{1.27cm}\=\hspace{1.27cm}\=\hspace{1.27cm}\=%
\hspace{1.27cm}\=\hspace{1.27cm}\=\hspace{1.27cm}\=\hspace{1.27cm}\=%
\hspace{1.27cm}\=\hspace{1.27cm}\=\kill
Name    \> \> Type   \> Description                                        \> \> \> \> \> \> \> Units\\
$-$$-$$-$$-$    \> \> $-$$-$$-$$-$   \> $-$$-$$-$$-$$-$$-$$-$$-$$-$$-$$-$                                        \> \> \> \> \> \> \> $-$$-$$-$$-$$-$\\
CGM\_ABORT\> \> L\> error message on database handling\> \> \> \> \> \> \> -\\
DELT\> \> R\> time step (= 1 day)\> \> \> \> \> \> \> d\\
EVSMX\> \> R\> maximum evaporation rate from a shaded soil surface\> \> \> \> \> \> \> cm d$^{{\rm -1}}$\\
EVST\> \> R\> total evaporation from a shaded soil surface\> \> \> \> \> \> \> cm\\
EVSTX\> \> R\> output cumulative water balance variable EVST\> \> \> \> \> \> \> cm\\
EVWMX\> \> R\> maximum evaporation rate from a shaded water surface\> \> \> \> \> \> \> cm d$^{{\rm -1}}$\\
EVWT\> \> R\> total evaporation from a shaded water surface\> \> \> \> \> \> \> cm\\
EVWTX\> \> R\> output cumulative water balance variable EVWT\> \> \> \> \> \> \> cm\\
IAIRDU\> \> I\> flag for airducts in root zone; (0) absent, (1) present\> \> \> \> \> \> \> -\\
ILSM\> \> I\> number of elements in table SMTAB\> \> \> \> \> \> \> -\\
ITASK\> \> I\> flag to control task to be performed\> \> \> \> \> \> \> -\\
ITOLD\> \> I\> retains number of last task performed\> \> \> \> \> \> \> -\\
IULOG\> \> I\> unit number, log file\> \> \> \> \> \> \> -\\
IUOUT\> \> I\> unit number, output file\> \> \> \> \> \> \> -\\
IUSO\> \> I\> unit number, soil data file\> \> \> \> \> \> \> -\\
SM\> \> R\> actual soil moisture content\> \> \> \> \> \> \> cm$^{{\rm 3}}$ cm$^{{\rm -3}}$\\
SM0\> \> R\> soil porosity; saturated moisture content\> \> \> \> \> \> \> cm$^{{\rm 3}}$ cm$^{{\rm -3}}$\\
SMFCF\> \> R\> soil moisture content at field capacity\> \> \> \> \> \> \> cm$^{{\rm 3}}$ cm$^{{\rm -3}}$\\
SMTAB\> \> R\> table with soil moisture data as a function of pF (AFGEN table)\> \> \> \> \> \> \> -\\
SMW\> \> R\> soil moisture content at wilting point\> \> \> \> \> \> \> cm$^{{\rm 3}}$ cm$^{{\rm -3}}$\\
SPG\_NO\> \> I\> soil group number\> \> \> \> \> \> \> -\\
SOFILE\> \> C\> name of soil data file\> \> \> \> \> \> \> -\\
TRA\> \> R\> actual transpiration rate (as calculated in EVTRA)\> \> \> \> \> \> \> cm d$^{{\rm -1}}$\\
TRAJPP\> \> R\> subtotal potential transpiration, potential production\\
TRAT\> \> R\> total transpiration\> \> \> \> \> \> \> cm\\
TRATX\> \> R\> output cumulative water balance variable TRAT\> \> \> \> \> \> \> cm\\
WATPP\> \> -\> COMMON BLOCK; summary output variables\> \> \> \> \> \> \> -
\end{tabbing}
\newpage
{\large {\bf Functions}}

\bigskip
\nwln
\begin{tabbing}
\hspace{1.27cm}\=\hspace{1.27cm}\=\hspace{1.27cm}\=\hspace{1.27cm}\=%
\hspace{1.27cm}\=\hspace{1.27cm}\=\hspace{1.27cm}\=\hspace{1.27cm}\=%
\hspace{1.27cm}\=\hspace{1.27cm}\=\kill
{\bf FUNCTION AFGEN} \> \> \> (TABLE, ILTAB, X)
\end{tabbing}
\nwln
\begin{tabbing}
\hspace{1.27cm}\=\hspace{1.27cm}\=\hspace{1.27cm}\=\hspace{1.27cm}\=%
\hspace{1.27cm}\=\hspace{1.27cm}\=\hspace{1.27cm}\=\hspace{1.27cm}\=%
\hspace{1.27cm}\=\hspace{1.27cm}\=\kill
Name    \> \> Type   \> Description                                        \> \> \> \> \> \> \> Units\\
$-$$-$$-$$-$    \> \> $-$$-$$-$$-$   \> $-$$-$$-$$-$$-$$-$$-$$-$$-$$-$$-$                                        \> \> \> \> \> \> \> $-$$-$$-$$-$$-$\\
AFGEN   \> \> R   \> returned value, result of interpolation            \> \> \> \> \> \> \> units Y\\
I       \> \> I   \> DO$-$loop control variable                           \> \> \> \> \> \> \> $-$\\
ILTAB\> \> I\> length of table\> \> \> \> \> \> \> -\\
SLOPE   \> \> R   \> slope of function in range containing X            \> \> \> \> \> \> \> units Y/X\\
TABLE   \> \> R   \> table name\> \> \> \> \> \> \> units X Y\\
X       \> \> R   \> value of independent variable, for which function is evaluated\> \> \> \> \> \> \> units X
\end{tabbing}

\bigskip
\nwln
\begin{tabbing}
\hspace{1.27cm}\=\hspace{1.27cm}\=\hspace{1.27cm}\=\hspace{1.27cm}\=%
\hspace{1.27cm}\=\hspace{1.27cm}\=\hspace{1.27cm}\=\hspace{1.27cm}\=%
\hspace{1.27cm}\=\hspace{1.27cm}\=\kill
{\bf FUNCTION LIMIT}\> \> \> ( )
\end{tabbing}
\nwln
\begin{tabbing}
\hspace{1.27cm}\=\hspace{1.27cm}\=\hspace{1.27cm}\=\hspace{1.27cm}\=%
\hspace{1.27cm}\=\hspace{1.27cm}\=\hspace{1.27cm}\=\hspace{1.27cm}\=%
\hspace{1.27cm}\=\hspace{1.27cm}\=\kill
Name    \> \> Type   \> Description                                        \> \> \> \> \> \> \> Units\\
-$-$$-$$-$    \> \> $-$$-$$-$$-$   \> $-$$-$$-$$-$$-$$-$$-$$-$$-$$-$$-$                                        \> \> \> \> \> \> \> $-$$-$$-$$-$$-$\\
LIMIT   \> \> R   \> returned value, limited within range of bounds     \> \> \> \> \> \> \> units X\\
P1      \> \> R   \> lower bound imposed on X                           \> \> \> \> \> \> \> units X\\
P2      \> \> R   \> upper bound imposed on X                           \> \> \> \> \> \> \> units X\\
X       \> \> R   \> value to be evaluated                              \> \> \> \> \> \> \> units X
\end{tabbing}

\bigskip
\nwln
\begin{tabbing}
\hspace{1.27cm}\=\hspace{1.27cm}\=\hspace{1.27cm}\=\hspace{1.27cm}\=%
\hspace{1.27cm}\=\hspace{1.27cm}\=\hspace{1.27cm}\=\hspace{1.27cm}\=%
\hspace{1.27cm}\=\hspace{1.27cm}\=\kill
{\bf FUNCTION SWEAF}\> \> \> (ET0, CGNR)
\end{tabbing}
\nwln
\begin{tabbing}
\hspace{1.27cm}\=\hspace{1.27cm}\=\hspace{1.27cm}\=\hspace{1.27cm}\=%
\hspace{1.27cm}\=\hspace{1.27cm}\=\hspace{1.27cm}\=\hspace{1.27cm}\=%
\hspace{1.27cm}\=\hspace{1.27cm}\=\kill
Name    \> \> Type   \> Description                                        \> \> \> \> \> \> \> Units\\
$-$$-$$-$$-$    \> \> $-$$-$$-$$-$   \> $-$$-$$-$$-$$-$$-$$-$$-$$-$$-$$-$                                        \> \> \> \> \> \> \> $-$$-$$-$$-$$-$\\
A       \> \> R   \> constant                                           \> \> \> \> \> \> \> $-$\\
B       \> \> R   \> constant                                           \> \> \> \> \> \> \> d cm$^{{\rm -1}}$\\
CGNR    \> \> R   \> crop group number (from 1(= drought sensitive) to 5(= drought resistent))   \> \> \> \> \> \> \> $-$  \\
ET0     \> \> R   \> potential evapotranspiration rate                  \> \> \> \> \> \> \> cm d$^{{\rm -1}}$ \\
SWEAF   \> \> R   \> returned value, fraction of easily available soil water between field \\
\>\> \> capacity and wilting point                                      \> \> \> \> \> \> \> $-$
\end{tabbing}

