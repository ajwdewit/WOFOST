
\chapter{Introduction} 

\section{About WOFOST}

WOFOST is the acronym for WOrld FOod STudies. It is the name of a model for
simulating the growth of crops. WOFOST has been continuously developed by
Wageningen Research and Wageningen University (and its predecessors) already
since the 1980-ies. The model is maintained by Wageningen 
Research and several implementation of the model are available which have been
developed for both research and operational applications.

for a generic introduction and overview to the WOFOST model see:\\
Wit, Allard de, Hendrik Boogaard, Davide Fumagalli, Sander Janssen, Rob Knapen, 
Daniel van Kraalingen, Iwan Supit, Raymond van der Wijngaart, and Kees van Diepen. 
\textit{25 Years of the WOFOST Cropping Systems Model.} Agricultural Systems 168 
(January 1, 2019): 154–67. \footnote{https://doi.org/10.1016/j.agsy.2018.06.018}


\section{Levels of crop production}
Three levels of plant
production are distinguished. The crop production systems at any of these levels can
be considered as members of a broad class. In order of decreasing yield, these levels
are (Penning de Vries et al. 1989):

{\it Potential yield level\/}\\
This is the potential production situation. The crop has ample water and nutrients.
Crop yield depends on the initial conditions, weather (temperature and radiation)
and crop features (mainly growing season length).
Dry matter production, in case of full canopy, amounts 150-350 kg ha$^{{\rm -1}}$d$^{{\rm -1}}$. This
production level will be reached in laboratory experiments, glasshouses and with
intensive farming systems.

{\it Attainable yield level\/}\\
In this scenario the yield of the crop is limited by the availability of water and/or nutrients 
during a part or the complete
growing season. In such a scenario the yield will be reduced but still crop management is assumed to be 
optimal.

{\it actual yield level\/}\\
In this scenario the crop yield can be further reduced by factors such as pest, disease, competition with weeds
or polutants such as ozone, salt or heavy metals. This yield level reflects what farmers actual
harvest from their fields. The gap between the different yield levels (yield gap) varies widely
across the globe and finding approaches to close the yield gap is a an area of intense research,
see http://yieldgap.org

Note that the WOFOST 7.1 crop simulation model can be applied in the domain of potential crop
production and production with a water shortage (Potential and attainable yield levels). 

\section{Guide to this manual}

This manual covers a detailed description of the processes of crop growth and water movement as they are
implemented in WOFOST Version 7.1. First of all, ancillary calculation will be described including:
\begin{itemize}
	\item derived meteorological variables
	\item reference evapotranspiration
	\item day length and solar elevation
	\item extra terrestrial radiation
\end{itemize}

Next, the different components of the crop simulation itself will be described in detail:
\begin{itemize}
	\item phenology
	\item transpiration
	\item assimilation:
	\subitem gross photosynthesis rate
	\subitem Correction for suboptimal temperature
	\subitem Correction for water stress
	\item Maintenance respiration
	\item growth of the crop:
	\subitem Net photosynthesis rate
	\subitem growth respiration
	\subitem partitioning
	\subitem leaf growth and senescence 
	\subitem stems
	\subitem Roots
	\subitem storage organs	
\end{itemize}

Finally, the soil components are described which consist of several water balance implementations.