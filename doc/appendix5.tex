\section{  APPENDIX 5  }

\bigskip
{\bf {\large Versions of WO\-FOST 6.0}}

Two versions of WOFOST 6.0 exist. One is the version currently implemented in the
Crop Growth Monitoring System of the Joint Research Centre of the EC (Ispra, Italy) and
a second one is a more general version. The main routine in the JRC version, SIMCRO,
as well as the driving and organizing routines, are partly written in SQL. These SQL
statements replace the data manipula\-tion routines in the general version, which are part of
the TTUTIL FORTRAN library. The general version is completely written in FOR\-TRAN.

Another important difference is the fact that the JRC version simulates crop growth on a
continental scale, whereas the general version simulates crop growth for one station only.
The grid which WOFOST JRC version uses, consists of 1350 cells of 50x50 kilometers
and covers the whole of the EC. This means that the input data of the JRC ver\-{\nobreak}sion,
weather-, plant-, and soil-data, refer to grid cells. This means, on its turn, that the input
data first have to be manipulated to grid data before they can be used as input in the
WOFOST model. The data manipulation of input and output is performed in ORACLE.
This explains why SQL statements are used. The simulation results should also be
interpreted on gridlevel. Bec\-ause of the use of SQL in the JRC version, and the different
scale on which this version is used, it is clear that the main routine, the routines which
handle and manipulate the data (meteorological, plant and soil) as well as the routines
which organize the simulation, are different from the driving and organizing routines in
the general version. See table A1 and table A2. 

The routines which take care of the calculation of evapotranspiration and the simulation of
crop growth, only differ in the way the input/output data are manipulat\-ed. The actual
simulation calculations are the same in both versions. These routines are {\bf ASSIM},
{\bf ASTRO}, {\bf CROPSI} (general version) or {\bf CRSIM} (JRC version), {\bf EVTRA}, {\bf PENMAN},
{\bf ROOTD}, {\bf TOTASS}, {\bf SWEAF}, {\bf WATPP}, {\bf WATFD} and {\bf WATGW}. They are discussed in
the chapters three, four and five. 

It should be mentioned, that the JRC version does not contain a weather generator nor
does it contain subrou\-tines which calculate the statistics of the potential produc\-tion or the
water limited production. The subroutine which calculates the nutrient requirement and
nutrient limited production is also omitted in the JRC version.

Furthermore, it should be noted that the global radiation term, in the subroutine {\bf PEN\-MAN}, in both versions can be obtained, using the \AA ngstr\"{o}m equation. In the general
version the empirical coefficients of this equation should be provided by the user.
However, in the JRC version, these coefficients are estimated (subroutine {\bf METEO}). 

In order to use the \AA ngstr\"{o}m equation, the sunshine duration should be known. If this
parame\-ter is not available, in the subsequent JRC versions, the global radiation will be
estimated using either the equation developed by Supit (1994) or the Hargreaves formula
(1985). The method developed by Supit uses cloud cover and maximum and minimum
temperature as input. The accuracy of the estimates is slightly less in comparison to the
results obtained with the \AA ngstr\"{o}m formula. The Hargreaves formula uses maximum and
minimum temperature only. The accuracy of this method is less then the accuracy of the
two earlier mentioned methods.

\bigskip
{\large {\bf Functions and subroutines used in WOFOST 6.0 (JRC version)}}

In table A1 the call structure of WOFOST 6.0 (JRC version) is depicted. Only the non
standard subroutines are presented. The standard FORTRAN function such as SIN and
COS have been omitted from the structure to make interpretation more easy. 

The functionality of all subroutines and functions is briefly described. This description is
split in two parts:\nwln
\begin{tabbing}
\hspace{1.27cm}\=\hspace{1.27cm}\=\hspace{1.27cm}\=\hspace{1.27cm}\=%
\hspace{1.27cm}\=\hspace{1.27cm}\=\hspace{1.27cm}\=\hspace{1.27cm}\=%
\hspace{1.27cm}\=\hspace{1.27cm}\=\kill
$\bullet$\> Non standard FORTRAN routines especially made for WOFOST 6.0;
\end{tabbing}
$\bullet$ 
\testlastline

\begin{indenting}{1.27cm}
Non standard FORTRAN routines from the library TTUTIL (Rappoldt {\it et al.\/}
1990);
\end{indenting}

Table A1 Call structure of subroutines and functions of WOFOST 6.0 (JRC version)

\strut\hfill {\bf [ Run specification]}
{\bf ...............................................................................................................................................................................................\-}
\begin{tabbing}
\hspace{1.27cm}\=\hspace{1.27cm}\=\hspace{1.27cm}\=\hspace{1.27cm}\=%
\hspace{1.27cm}\=\hspace{1.27cm}\=\hspace{1.27cm}\=\hspace{1.27cm}\=%
\hspace{1.27cm}\=\hspace{1.27cm}\=\kill
SIMCRO\> ORA\_ERR\> \> CGM\_ORA\_ERR\\
\>DLYLD\\
\>DAYS\_BETWEEN\\
\>PARAM\_CORRECT
\end{tabbing}
\strut\hfill {\bf [ Loop over years ]}\\
 \GrBox(1010)\GrBox(1010)\GrBox(1010)\GrBox(1010)\GrBox(1010)\GrBox(1010)\GrBox(1010)\GrBox(1010)\GrBox(1010)\GrBox(1010)\GrBox(1010)\GrBox(1010)\GrBox(1010)\GrBox(1010)\GrBox(1010)\GrBox(1010)\GrBox(1010)\GrBox(1010)\GrBox(1010)\GrBox(1010)\GrBox(1010)\GrBox(1010)\GrBox(1010)\GrBox(1010)\GrBox(1010)\GrBox(1010)\GrBox(1010)\GrBox(1010)\GrBox(1010)\GrBox(1010)\GrBox(1010)\GrBox(1010)\GrBox(1010)\GrBox(1010)\GrBox(1010)\GrBox(1010)\GrBox(1010)\GrBox(1010)\GrBox(1010)\GrBox(1010)\GrBox(1010)\GrBox(1010)\GrBox(1010)\GrBox(1010)\GrBox(1010)\GrBox(1010)\GrBox(1010)\GrBox(1010)\GrBox(1010)\GrBox(1010)\GrBox(1010)\GrBox(1010)\GrBox(1010)\GrBox(1010)\GrBox(1010)\GrBox(1010)\GrBox(1010)\GrBox(1010)\GrBox(1010)\GrBox(1010)\GrBox(1010)\GrBox(1010)\GrBox(1010)\nwln
\begin{tabbing}
\hspace{1.27cm}\=\hspace{1.27cm}\=\hspace{1.27cm}\=\hspace{1.27cm}\=%
\hspace{1.27cm}\=\hspace{1.27cm}\=\hspace{1.27cm}\=\hspace{1.27cm}\=%
\hspace{1.27cm}\=\hspace{1.27cm}\=\kill
\>WOF60\> \> DCRPST\> \> DATES\> \> ERROR\\
\>\> \> \> \> CGM\_ERROR\\
\>\> \> \> \> ORA\_ERR
\end{tabbing}
\strut\hfill {\bf [ Loop over days ]}\\
 \GrBox(2020)\GrBox(2020)\GrBox(2020)\GrBox(2020)\GrBox(2020)\GrBox(2020)\GrBox(2020)\GrBox(2020)\GrBox(2020)\GrBox(2020)\GrBox(2020)\GrBox(2020)\GrBox(2020)\GrBox(2020)\GrBox(2020)\GrBox(2020)\GrBox(2020)\GrBox(2020)\GrBox(2020)\GrBox(2020)\GrBox(2020)\GrBox(2020)\GrBox(2020)\GrBox(2020)\GrBox(2020)\GrBox(2020)\GrBox(2020)\GrBox(2020)\GrBox(2020)\GrBox(2020)\GrBox(2020)\GrBox(2020)\GrBox(2020)\GrBox(2020)\GrBox(2020)\GrBox(2020)\GrBox(2020)\GrBox(2020)\GrBox(2020)\GrBox(2020)\GrBox(2020)\GrBox(2020)\GrBox(2020)\GrBox(2020)\GrBox(2020)\GrBox(2020)\GrBox(2020)\GrBox(2020)\GrBox(2020)\GrBox(2020)\GrBox(2020)\GrBox(2020)\GrBox(2020)\GrBox(2020)\GrBox(2020)\GrBox(2020)\GrBox(2020)\GrBox(2020)\GrBox(2020)\GrBox(2020)\GrBox(2020)\GrBox(2020)\GrBox(2020)\nwln
\begin{tabbing}
\hspace{1.27cm}\=\hspace{1.27cm}\=\hspace{1.27cm}\=\hspace{1.27cm}\=%
\hspace{1.27cm}\=\hspace{1.27cm}\=\hspace{1.27cm}\=\hspace{1.27cm}\=%
\hspace{1.27cm}\=\hspace{1.27cm}\=\kill
\>\> \> WOFSIM\> \> TIMER\> \> ERROR\\
\>\> \> \> \> METEO\> \> CGM\_ORA\_ERR\\
\>\> \> \> \> \> \> PENMAN\\
\>\> \> \> \> STDAY\> \> CGM\_ORA\_ERR\\
\>\> \> \> \> \> \> AFGEN\\
\>\> \> \> \> CRSIM\> \> AFGEN\\
\>\> \> \> \> \> \> LIMIT\\
\>\> \> \> \> \> \> ASTRO\\
\>\> \> \> \> \> \> TOTASS\> \> ASSIM\\
\>\> \> \> \> \> \> EVTRA\> \> SWEAF\> LIMIT\\
\>\> \> \> \> \> \> \> \> LIMIT\\
\>\> \> \> \> \> \> CGM\_ERROR\\
\>\> \> \> \> ROOTD\\
\>\> \> \> \> CGM\_ERR\\
\>\> \> \> \> WATPP\\
\>\> \> \> \> WATFD\> \> LIMIT\\
\>\> \> \> \> \> \> AFGEN\\
\>\> \> \> \> WATGW\> \> LIMIT\\
\>\> \> \> \> \> \> AFGEN\\
\>\> \> \> \> \> \> SUBSOL\\
\>\> \> \> \> DCRPST\\
\>\> \> \> \> PRIWGW\> \> PAUZE
\end{tabbing}
\strut\hfill {\bf [ Loop over days ]}\\
 \GrBox(2020)\GrBox(2020)\GrBox(2020)\GrBox(2020)\GrBox(2020)\GrBox(2020)\GrBox(2020)\GrBox(2020)\GrBox(2020)\GrBox(2020)\GrBox(2020)\GrBox(2020)\GrBox(2020)\GrBox(2020)\GrBox(2020)\GrBox(2020)\GrBox(2020)\GrBox(2020)\GrBox(2020)\GrBox(2020)\GrBox(2020)\GrBox(2020)\GrBox(2020)\GrBox(2020)\GrBox(2020)\GrBox(2020)\GrBox(2020)\GrBox(2020)\GrBox(2020)\GrBox(2020)\GrBox(2020)\GrBox(2020)\GrBox(2020)\GrBox(2020)\GrBox(2020)\GrBox(2020)\GrBox(2020)\GrBox(2020)\GrBox(2020)\GrBox(2020)\GrBox(2020)\GrBox(2020)\GrBox(2020)\GrBox(2020)\GrBox(2020)\GrBox(2020)\GrBox(2020)\GrBox(2020)\GrBox(2020)\GrBox(2020)\GrBox(2020)\GrBox(2020)\GrBox(2020)\GrBox(2020)\GrBox(2020)\GrBox(2020)\GrBox(2020)\GrBox(2020)\GrBox(2020)\GrBox(2020)\GrBox(2020)\GrBox(2020)\GrBox(2020)\\
\strut\hfill {\bf [ Loop over years ]}\\
 \GrBox(1010)\GrBox(1010)\GrBox(1010)\GrBox(1010)\GrBox(1010)\GrBox(1010)\GrBox(1010)\GrBox(1010)\GrBox(1010)\GrBox(1010)\GrBox(1010)\GrBox(1010)\GrBox(1010)\GrBox(1010)\GrBox(1010)\GrBox(1010)\GrBox(1010)\GrBox(1010)\GrBox(1010)\GrBox(1010)\GrBox(1010)\GrBox(1010)\GrBox(1010)\GrBox(1010)\GrBox(1010)\GrBox(1010)\GrBox(1010)\GrBox(1010)\GrBox(1010)\GrBox(1010)\GrBox(1010)\GrBox(1010)\GrBox(1010)\GrBox(1010)\GrBox(1010)\GrBox(1010)\GrBox(1010)\GrBox(1010)\GrBox(1010)\GrBox(1010)\GrBox(1010)\GrBox(1010)\GrBox(1010)\GrBox(1010)\GrBox(1010)\GrBox(1010)\GrBox(1010)\GrBox(1010)\GrBox(1010)\GrBox(1010)\GrBox(1010)\GrBox(1010)\GrBox(1010)\GrBox(1010)\GrBox(1010)\GrBox(1010)\GrBox(1010)\GrBox(1010)\GrBox(1010)\GrBox(1010)\GrBox(1010)\GrBox(1010)\GrBox(1010)\\
\strut\hfill {\bf [ Run specification]}\\
{\bf ...............................................................................................................................................................................................\-}\\
 {\bf {\large WOFOST specific subroutines and functions (JRC version)}}

AFGEN
\testlastline

\begin{indenting}{5.08cm}
Function which is used for linear interpola\-tion in a one-di\-mensi\-onal array with paired data.
\end{indenting}
ASSIM
\testlastline

\begin{indenting}{5.08cm}
Subroutine which calculates the gross CO$_{{\rm 2}}$ assimilation rate
of a crop, using Gaussian integration over depth in the
canopy. The assimilation rate is computed for given fluxes of
photosyntheti\-cally active radiation, whereafter integration
over depth takes place.
\end{indenting}
ASTRO
\testlastline

\begin{indenting}{5.08cm}
Subroutine which calculates astronomic day length, diurnal
radia\-tion characteristics such as atmospheric transmission,
diffuse radiation etc.
\end{indenting}
CGM\_ERR
\testlastline

\begin{indenting}{5.08cm}
Notifies user (at terminal) of an error during SQL, stops if
error is fatal.
\end{indenting}
CRSIM
\testlastline

\begin{indenting}{5.08cm}
Subroutine which simulates the potential or water limited
crop growth.
\end{indenting}
DATES
\testlastline

\begin{indenting}{5.08cm}
Converts either a year and day number to a month num\-ber
and a day number within that month (ACTI\-ON='TO\-{\nobreak}\_MONTHS') or the inverse, to a day number within in the
year (ACTI\-{\nobreak}ON='TO\_YEAR'). Leap years are recognized.
\end{indenting}
DAYS\_BETWEEN
\testlastline

\begin{indenting}{5.08cm}
Calculates the absolute number of days between date 1 and
date 2.
\end{indenting}
DCRPST
\testlastline

\begin{indenting}{5.08cm}
Calculates growth statistics for a period of one decade, (a
month has three decades, days: 1-10, 11-20, 21-28, 29, 30,\-
31).
\end{indenting}
DLYLD
\testlastline

\begin{indenting}{5.08cm}
Deletes any record from SIM\_YIELD and CROP\_YIELD in
the ORACLE database that have been filled during an unfin\-ished simulati\-on run.
\end{indenting}
EVTRA
\testlastline

\begin{indenting}{5.08cm}
Subroutine which calculates for a given crop cover the maxi\-mum evaporation rate  from a shaded wet soil surfa\-ce and
from a shaded water surface. It also calculates the maximum
and actual crop transpiration rate.
\end{indenting}
METEO
\testlastline

\begin{indenting}{5.08cm}
Subroutine which provides the calling program with weat\-her
data on grid. This includes rainfall and evapotranspi\-ra\-tion.
\end{indenting}
ORA\_ERR
\testlastline

\begin{indenting}{5.08cm}
Notifies user (at terminal) of an error during SQL, skips "not
found" warning.
\end{indenting}
PENMAN
\testlastline

\begin{indenting}{5.08cm}
Subroutine which calculates the potential evapo(trans\-pi)ration
rates from a free water surface, a bare soil surface and a
crop canopy. The method developed by Penman is used.
\end{indenting}
\begin{tabbing}
\hspace{1.27cm}\=\hspace{1.27cm}\=\hspace{1.27cm}\=\hspace{1.27cm}\=%
\hspace{1.27cm}\=\hspace{1.27cm}\=\hspace{1.27cm}\=\hspace{1.27cm}\=%
\hspace{1.27cm}\=\hspace{1.27cm}\=\kill
PAUZE\> \> \> \> Wait for $<$return$>$ from keyboard.
\end{tabbing}
PARAM\_CORRECT
\testlastline

\begin{indenting}{5.08cm}
Corrects the values of IENCHO, IDAYEN and IDUR\-MX,
based on crop type.
\end{indenting}
PRIWGW
\testlastline

\begin{indenting}{5.08cm}
Prints part of the standard output table, WOFOST.OUT
showing variables of crop growth and waterbalance at inter\-val at a predefined print interval. The waterlimited produc\-tion is considered here without groundwater influence.
\end{indenting}

 \bigskip
 ROOTD
\testlastline

\begin{indenting}{5.08cm}
Subroutine which calculates the depth of the root for each
day of the crop cycle.
\end{indenting}
SIMCRO
\testlastline

\begin{indenting}{5.08cm}
Main procedure of the crop growth\-{\nobreak} simulati\-on model.
\end{indenting}
STDAY
\testlastline

\begin{indenting}{5.08cm}
Determines the start day if a variable sowing date is chosen.
\end{indenting}
SUBSOL
\testlastline

\begin{indenting}{5.08cm}
Subroutine which calculates the rate of capillary flow or
perco\-lation between ground water table and root zone. The
stationary flow is found by Gaussian integrati\-on. In an
iteration loop the correct flow is found. The integrati\-on goes
over four intervals at most.
\end{indenting}
SWEAF
\testlastline

\begin{indenting}{5.08cm}
This function describes the relati\-onship given in tabular form
by Dooren\-bos \& Kas\-sam (1979) and by Van Keulen \& Wolf
(1986) of the fraction of easily available soil water between
field capacity and wilting point as a function of the potential
evapotranspirati\-on rate (for a closed canopy) in cm d$^{{\rm -1}}$ and
the crop group number (from 1 (=drought-sensitive) to 5
(=drought-resistent)).
\end{indenting}
TOTASS
\testlastline

\begin{indenting}{5.08cm}
Subroutine which calculates the daily gross CO$_{{\rm 2}}$ assimilati\-on
by per\-forming a Gaussian integration over time. At three
different times of the day, irradiance is computed and used
to calculate the instantaneous canopy assimilati\-on, whereafter
integration takes place. 
\end{indenting}
WATFD
\testlastline

\begin{indenting}{5.08cm}
Subroutine which keeps track of the soil water balance for
freely draining soils in the water limited production situa\-tion.
\end{indenting}
WATGW
\testlastline

\begin{indenting}{5.08cm}
Subroutine which simulates the soil water balance is per\-formed for soils influenced by the presence of groun\-dwater.
Two situati\-ons are distin\-guished: with or without artificial
drainage. The soil water balance is calculated for a cropped
field in the water-limited production situation.
\end{indenting}
WATPP
\testlastline

\begin{indenting}{5.08cm}
Subroutine which computes the variables of the soil water
balan\-ce in the potential production situation. The purpose is
to quantify the crop requirements for continuous growth
without drought stress. It is assu\-med that the soil is per\-manently at field capacity.
\end{indenting}
WOF60
\testlastline

\begin{indenting}{5.08cm}
Driving routine for the simulation routine WOFSIM. Takes
care of run control, input/output options, the call of the
potential and water limited run and the call of the years
specific calculations.
\end{indenting}
WOFSIM
\testlastline

\begin{indenting}{5.08cm}
This routine organizes the simulation of crop growth and of
soil water balance conditions and the writing of reports by
successive calls to the relevant weather, crop, soil water and
printing subroutines in that order.
\end{indenting}

\bigskip
\bigskip
\bigskip
\bigskip
\bigskip
{\bf {\large Subroutines and functions from FORTRAN utility library TTUTIL}}

ERROR
\testlastline

\begin{indenting}{5.08cm}
Writes an error message to the screen and holds the screen
until $<$RE\-TURN$>$ is pressed.\hfill  
\end{indenting}
LIMIT
\testlastline

\begin{indenting}{5.08cm}
Returns value of X limited within a specified interval.\hfill  
\end{indenting}
TIMER
\testlastline

\begin{indenting}{5.08cm}
Subroutine which updates the time (from the start of the
simula\-tion) and related variables each time it is called with
ITASK=2. It will set a flag to .TRUE. if the finish time of
the simulation is reached (counted from start of simulation).
The routine is initialized first by a call with ITASK=1. The
first six arguments will then be made local. Leap years are
handled correctly.\hfill  
\end{indenting}

\bigskip
\bigskip
\bigskip
{\bf {\large Functions and subroutines used in WOFOST 6.0 (general version)}}

In table A2 the call structure of WOFOST 6.0 general version is depicted. Only the non
standard subroutines and functions are depicted. The standard FOR\-TRAN functions like
SIN and COS have been omitted from the structure to make interpre\-tation more easy. The
functionality of all subroutines and functions is described. This description is split in three
parts:\nwln
\begin{tabbing}
\hspace{1.27cm}\=\hspace{1.27cm}\=\hspace{1.27cm}\=\hspace{1.27cm}\=%
\hspace{1.27cm}\=\hspace{1.27cm}\=\hspace{1.27cm}\=\hspace{1.27cm}\=%
\hspace{1.27cm}\=\hspace{1.27cm}\=\kill
$\bullet$\> Non standard FORTRAN routines especially made for WOFOST 6.0;
\end{tabbing}
$\bullet$ 
\testlastline

\begin{indenting}{1.27cm}
Non standard FORTRAN routines from the library TTUTIL (Rappoldt {\it et al\/}.
1990);
\end{indenting}

\bigskip
Table A2 Call structure of subroutines and functions of WOFOST Version 6.0

\strut\hfill {\bf [ Run specification ]}\\
...............................................................................................................................................................................................\-\nwln
\begin{tabbing}
\hspace{1.27cm}\=\hspace{1.27cm}\=\hspace{1.27cm}\=\hspace{1.27cm}\=%
\hspace{1.27cm}\=\hspace{1.27cm}\=\hspace{1.27cm}\=\hspace{1.27cm}\=%
\hspace{1.27cm}\=\hspace{1.27cm}\=\kill
W60MAIN\> ILEN\\
\>\> RDINIT\> \> RDDATA\> \> ILEN\\
\>\> \> \> IFINDC\> \> ERROR\\
\>\> \> \> UPPERC\\
\>\> \> \> EXTENS\> \> ILEN\\
\>\> \> \> \> \> ERROR\\
\>\> \> \> \> \> UPPERC\\
\>\> \> \> RDINDX\> \> ILEN\\
\>\> \> \> \> \> IFINDC\\
\>\> \> \> \> \> ERROR\\
\>\> \> \> \> \> FOPENG\> \> ILEN\\
\>\> \> \> \> \> \> \> ERROR\\
\>\> \> \> \> \> \> \> UPPERC\\
\>\> \> \> \> \> \> \> LOWERC\\
\>\> \> \> \> \> UPPERC\\
\>\> \> \> ERROR\\
\>\> \> \> FOPENG\\
\>\> RDSCHA\> \> RDSCHA\\
\>\> ERROR\\
\>\> ENTDIN\> \> ILEN\\
\>\> \> \> ISTART
\end{tabbing}

\bigskip
\strut\hfill {\bf [ Loop over years ]}\\
 \GrBox(1010)\GrBox(1010)\GrBox(1010)\GrBox(1010)\GrBox(1010)\GrBox(1010)\GrBox(1010)\GrBox(1010)\GrBox(1010)\GrBox(1010)\GrBox(1010)\GrBox(1010)\GrBox(1010)\GrBox(1010)\GrBox(1010)\GrBox(1010)\GrBox(1010)\GrBox(1010)\GrBox(1010)\GrBox(1010)\GrBox(1010)\GrBox(1010)\GrBox(1010)\GrBox(1010)\GrBox(1010)\GrBox(1010)\GrBox(1010)\GrBox(1010)\GrBox(1010)\GrBox(1010)\GrBox(1010)\GrBox(1010)\GrBox(1010)\GrBox(1010)\GrBox(1010)\GrBox(1010)\GrBox(1010)\GrBox(1010)\GrBox(1010)\GrBox(1010)\GrBox(1010)\GrBox(1010)\GrBox(1010)\GrBox(1010)\GrBox(1010)\GrBox(1010)\GrBox(1010)\GrBox(1010)\GrBox(1010)\GrBox(1010)\GrBox(1010)\GrBox(1010)\GrBox(1010)\GrBox(1010)\GrBox(1010)\GrBox(1010)\GrBox(1010)\GrBox(1010)\GrBox(1010)\GrBox(1010)\GrBox(1010)\GrBox(1010)\GrBox(1010)\nwln
\begin{tabbing}
\hspace{1.27cm}\=\hspace{1.27cm}\=\hspace{1.27cm}\=\hspace{1.27cm}\=%
\hspace{1.27cm}\=\hspace{1.27cm}\=\hspace{1.27cm}\=\hspace{1.27cm}\=%
\hspace{1.27cm}\=\hspace{1.27cm}\=\kill
\>\> WOF60\> \> ILEN\\
\>\> \> \> ISTART\\
\>\> \> \> ENTDIN\\
\>\> \> \> LOWERC\\
\>\> \> \> FOPENG\\
\>\> \> \> ENTDCH\> \> ILEN\\
\>\> \> \> \> \> ISTART\\
\>\> \> \> RDSETS\> \> ILEN\\
\>\> \> \> \> \> RDDATA\\
\>\> \> \> \> \> AMBUSY\> \> IFINDC\\
\>\> \> \> \> \> \> \> UPPERC\\
\>\> \> \> \> \> \> \> ERROR\\
\>\> \> \> ERROR\\
\>\> \> \> RDFROM\> \> RDDATA\\
\>\> \> \> \> \> AMBUSY\\
\>\> \> \> RDINIT\\
\>\> \> \> RDSCHA\\
\>\> \> \> RDSINT\> \> RDDATA\\
\>\> \> \> RDSREA\> \> RDDATA\\
\>\> \> \> MENU\> \> ISTART\\
\>\> \> \> \> \> ILEN\\
\>\> \> \> \> \> FOPENG\\
\>\> \> \> \> \> GETREC\\
\>\> \> \> \> \> WORDS\\
\>\> \> \> \> \> DECINT\\
\>\> \> \> \> \> ERROR\\
\>\> \> \> \> \> ENTCHA\\
\>\> \> \> MENMET\> \> ILEN\\
\>\> \> \> \> \> WORDS\\
\>\> \> \> \> \> ERROR\\
\>\> \> \> \> \> DECINT\\
\>\> \> \> \> \> ENTDIN\\
\>\> \> \> \> \> DECREA\> \> DECCHK\\
\>\> \> \> DECINT\\
\>\> \> \> MENRAN\> \> ILEN\\
\>\> \> \> \> \> WORDS\\
\>\> \> \> \> \> DECINT\\
\>\> \> \> \> \> ENTDIN\\
\>\> \> \> \> \> ERROR\\
\>\> \> \> UPPERC\\
\>\> \> \> ENTDRE\> \> ILEN\\
\>\> \> \> SELOUT\> \> ILEN\\
\>\> \> \> \> \> ENTDCH\\
\>\> \> \> \> \> UPPERC\\
\>\> \> \> \> \> ENTDRE\\
\>\> \> \> \> \> ENTDIN\\
\>\> \> \> \> \> ERROR\\
\>\> \> \> \> \> LOWERC\\
\>\> \> \> \> \> FOPENG\\
\>\> \> \> PRHEAD\> \> ILEN\\
\>\> \> \> PAUZE
\end{tabbing}
\strut\hfill {\bf [ Loop over days ]}\\
 \GrBox(2020)\GrBox(2020)\GrBox(2020)\GrBox(2020)\GrBox(2020)\GrBox(2020)\GrBox(2020)\GrBox(2020)\GrBox(2020)\GrBox(2020)\GrBox(2020)\GrBox(2020)\GrBox(2020)\GrBox(2020)\GrBox(2020)\GrBox(2020)\GrBox(2020)\GrBox(2020)\GrBox(2020)\GrBox(2020)\GrBox(2020)\GrBox(2020)\GrBox(2020)\GrBox(2020)\GrBox(2020)\GrBox(2020)\GrBox(2020)\GrBox(2020)\GrBox(2020)\GrBox(2020)\GrBox(2020)\GrBox(2020)\GrBox(2020)\GrBox(2020)\GrBox(2020)\GrBox(2020)\GrBox(2020)\GrBox(2020)\GrBox(2020)\GrBox(2020)\GrBox(2020)\GrBox(2020)\GrBox(2020)\GrBox(2020)\GrBox(2020)\GrBox(2020)\GrBox(2020)\GrBox(2020)\GrBox(2020)\GrBox(2020)\GrBox(2020)\GrBox(2020)\GrBox(2020)\GrBox(2020)\GrBox(2020)\GrBox(2020)\GrBox(2020)\GrBox(2020)\GrBox(2020)\GrBox(2020)\GrBox(2020)\GrBox(2020)\GrBox(2020)\nwln
\begin{tabbing}
\hspace{1.27cm}\=\hspace{1.27cm}\=\hspace{1.27cm}\=\hspace{1.27cm}\=%
\hspace{1.27cm}\=\hspace{1.27cm}\=\hspace{1.27cm}\=\hspace{1.27cm}\=%
\hspace{1.27cm}\=\hspace{1.27cm}\=\kill
\>\> \> \> WOFSIM\> \> TIMER\> \> ERROR\\
\>\> \> \> \> \> METEO\> \> CLIMRD\> FOPENG\\
\>\> \> \> \> \> \> \> \> MOFLIP\\
\>\> \> \> \> \> \> \> \> ERROR\\
\>\> \> \> \> \> \> \> \> AFGEN\\
\>\> \> \> \> \> \> \> \> RNGEN \> RANDOM ERROR\\
\>\> \> \> \> \> \> \> \> \> GAMMA2 RANDOM\\
\>\> \> \> \> \> \> \> \> RNDIS\> RANDOM\\
\>\> \> \> \> \> \> \> \> \> ERROR\\
\>\> \> \> \> \> \> \> \> \> GAMMA2\\
\>\> \> \> \> \> \> \> STINFO\\
\>\> \> \> \> \> \> \> WEATHR\\
\>\> \> \> \> \> \> \> ERROR\\
\>\> \> \> \> \> \> \> REPRD\> IFINDC\\
\>\> \> \> \> \> \> \> \> LOWERC\\
\>\> \> \> \> \> \> \> \> RDMINF\> ILEN\\
\>\> \> \> \> \> \> \> \> \> LOWERC\\
\>\> \> \> \> \> \> \> \> \> FOPENG\\
\>\> \> \> \> \> \> \> \> \> ERROR\\
\>\> \> \> \> \> \> \> \> ERROR\\
\>\> \> \> \> \> \> \> \> FOPENG\\
\>\> \> \> \> \> \> \> \> DATES\> ERROR\\
\>\> \> \> \> \> \> \> \> CONVR2\> ERROR\\
\>\> \> \> \> \> \> \> RNREAL\> FOPENG\\
\>\> \> \> \> \> \> \> \> MOFLIP\\
\>\> \> \> \> \> \> \> \> ERROR\\
\>\> \> \> \> \> \> \> PENMAN\> LIMIT\> ERROR\\
\>\> \> \> \> \> \> \> \> ASTRO\\
\>\> \> \> \> \> STDAY\> \> RDINIT\\
\>\> \> \> \> \> \> \> RDSREA\\
\>\> \> \> \> \> \> \> AFGEN\\
\>\> \> \> \> \> \> \> ERROR\\
\>\> \> \> \> \> CROPSI\> \> RDINIT\\
\>\> \> \> \> \> \> \> RDSINT\\
\>\> \> \> \> \> \> \> RDSREA\\
\>\> \> \> \> \> \> \> RDAREA\> RDDATA\\
\>\> \> \> \> \> \> \> AFGEN\\
\>\> \> \> \> \> \> \> LIMIT\\
\>\> \> \> \> \> \> \> ASTRO\\
\>\> \> \> \> \> \> \> TOTASS\> ASSIM\\
\>\> \> \> \> \> \> \> EVTRA\> SWEAF\> LIMIT\\
\>\> \> \> \> \> \> \> \> LIMIT\\
\>\> \> \> \> \> \> \> ERROR\\
\>\> \> \> \> \> ERROR\\
\>\> \> \> \> \> ROOTD\\
\>\> \> \> \> \> WATPP\> \> RDINIT\\
\>\> \> \> \> \> RDSREA\\
\>\> \> \> \> \> WATFD\> \> RDINIT\\
\>\> \> \> \> \> \> \> RDSREA\\
\>\> \> \> \> \> \> \> LIMIT\\
\>\> \> \> \> \> \> \> AFGEN\\
\>\> \> \> \> \> WATGW\> \> RDINIT\\
\>\> \> \> \> \> \> \> RDSREA\\
\>\> \> \> \> \> \> \> RDAREA\\
\>\> \> \> \> \> \> \> AFGEN\\
\>\> \> \> \> \> \> \> LIMIT\\
\>\> \> \> \> \> \> \> SUBSOL\> AFGEN\\
\>\> \> \> \> \> PRIJRC\> \> ILEN\\
\>\> \> \> \> \> \> \> LOWERC\\
\>\> \> \> \> \> \> \> FOPENG\\
\>\> \> \> \> \> \> \> EXTENS\\
\>\> \> \> \> \> \> \> ERROR\\
\>\> \> \> \> \> PRIWPP\\
\>\> \> \> \> \> PRIWFD\> \> PAUZE\\
\>\> \> \> \> \> PRIWGW\> \> PAUZE
\end{tabbing}
\strut\hfill {\bf [ Loop over days ]}\\
 \GrBox(2020)\GrBox(2020)\GrBox(2020)\GrBox(2020)\GrBox(2020)\GrBox(2020)\GrBox(2020)\GrBox(2020)\GrBox(2020)\GrBox(2020)\GrBox(2020)\GrBox(2020)\GrBox(2020)\GrBox(2020)\GrBox(2020)\GrBox(2020)\GrBox(2020)\GrBox(2020)\GrBox(2020)\GrBox(2020)\GrBox(2020)\GrBox(2020)\GrBox(2020)\GrBox(2020)\GrBox(2020)\GrBox(2020)\GrBox(2020)\GrBox(2020)\GrBox(2020)\GrBox(2020)\GrBox(2020)\GrBox(2020)\GrBox(2020)\GrBox(2020)\GrBox(2020)\GrBox(2020)\GrBox(2020)\GrBox(2020)\GrBox(2020)\GrBox(2020)\GrBox(2020)\GrBox(2020)\GrBox(2020)\GrBox(2020)\GrBox(2020)\GrBox(2020)\GrBox(2020)\GrBox(2020)\GrBox(2020)\GrBox(2020)\GrBox(2020)\GrBox(2020)\GrBox(2020)\GrBox(2020)\GrBox(2020)\GrBox(2020)\GrBox(2020)\GrBox(2020)\GrBox(2020)\GrBox(2020)\GrBox(2020)\GrBox(2020)\GrBox(2020)\nwln
\begin{tabbing}
\hspace{1.27cm}\=\hspace{1.27cm}\=\hspace{1.27cm}\=\hspace{1.27cm}\=%
\hspace{1.27cm}\=\hspace{1.27cm}\=\hspace{1.27cm}\=\hspace{1.27cm}\=%
\hspace{1.27cm}\=\hspace{1.27cm}\=\kill
\>\> \> \> STATPP\> \> LOWERC\\
\>\> \> \> \> \> FOPENG\\
\>\> \> \> \> \> PRHEAD\\
\>\> \> \> STATWP\> \> LOWERC\\
\>\> \> \> \> \> FOPENG\\
\>\> \> \> \> \> PRHEAD\\
\>\> \> \> NUTRIE\> \> RDINIT\\
\>\> \> \> \> \> RDSREA\\
\>\> \> \> \> \> AFGEN
\end{tabbing}
\strut\hfill {\bf [ Loop over years ]}\\
 \GrBox(1010)\GrBox(1010)\GrBox(1010)\GrBox(1010)\GrBox(1010)\GrBox(1010)\GrBox(1010)\GrBox(1010)\GrBox(1010)\GrBox(1010)\GrBox(1010)\GrBox(1010)\GrBox(1010)\GrBox(1010)\GrBox(1010)\GrBox(1010)\GrBox(1010)\GrBox(1010)\GrBox(1010)\GrBox(1010)\GrBox(1010)\GrBox(1010)\GrBox(1010)\GrBox(1010)\GrBox(1010)\GrBox(1010)\GrBox(1010)\GrBox(1010)\GrBox(1010)\GrBox(1010)\GrBox(1010)\GrBox(1010)\GrBox(1010)\GrBox(1010)\GrBox(1010)\GrBox(1010)\GrBox(1010)\GrBox(1010)\GrBox(1010)\GrBox(1010)\GrBox(1010)\GrBox(1010)\GrBox(1010)\GrBox(1010)\GrBox(1010)\GrBox(1010)\GrBox(1010)\GrBox(1010)\GrBox(1010)\GrBox(1010)\GrBox(1010)\GrBox(1010)\GrBox(1010)\GrBox(1010)\GrBox(1010)\GrBox(1010)\GrBox(1010)\GrBox(1010)\GrBox(1010)\GrBox(1010)\GrBox(1010)\GrBox(1010)\GrBox(1010)\\
\strut\hfill {\bf [ Run specification]}\\
{\bf ...............................................................................................................................................................................................\-}

\bigskip
\bigskip
\bigskip
\bigskip
\bigskip
\bigskip
\bigskip
\bigskip
{\bf {\large WOFOST specific subroutines and functions (general version)}}

AFGEN
\testlastline

\begin{indenting}{2.54cm}
Function which is used for linear interpola\-tion in a one-di\-mensi\-onal array
with paired data.
\end{indenting}
ASSIM
\testlastline

\begin{indenting}{2.54cm}
This routine calculates the gross CO$_{{\rm 2}}$ assimilation rate of the whole crop,
by performing a Gaussian integration over depth in the canopy (i.e. for
different values of LAI). The assimilation rate is computed for given fluxes
of photosynthetically active radiation, whereafter integra\-tion over depth
takes place.
\end{indenting}
ASTRO
\testlastline

\begin{indenting}{2.54cm}
This subroutine calculates astronomic day length, diurnal radiation charac\-teristics such as atmospheric transmission, diffuse radiation etc. 
\end{indenting}
CLIMRD
\testlastline

\begin{indenting}{2.54cm}
This routine derives daily weather data from long term average monthly
weather data (the so called WOFOST format)
\end{indenting}
CONVR2
\testlastline

\begin{indenting}{2.54cm}
This routine converts weather data from 'repaired DBMETEO' data to
appropriate units and vice versa. 
\end{indenting}
CROPSI
\testlastline

\begin{indenting}{2.54cm}
In this subroutine the simulation of the potential or water limited crop
growth is performed.
\end{indenting}
DATES
\testlastline

\begin{indenting}{2.54cm}
Converts either year and day number to a month number and day number
or vice versa.
\end{indenting}
EVTRA
\testlastline

\begin{indenting}{2.54cm}
This routine calculates for a given crop cover the maximum evaporation
rate from a shaded wet soil surface and from a shaded water surface and
the maximum and actual crop transpiration rate.
\end{indenting}
FOPENG
\testlastline

\begin{indenting}{2.54cm}
Opens a sequential, formatted file after doing an inquiry about the exis\-tence. The FOPENG used in WOFOST Version 6.0 is different from the
FOPENG routine available in the TTUTIL library.
\end{indenting}
GAMMA2
\testlastline

\begin{indenting}{2.54cm}
Generates a gamma distributed pseudo random variate. The gamma distri\-bution has two parameters, ALFA and BETA. This generator works for
0.0 $<$ ALFA $<$= 1.0 only. The algorithm used is derived from that of
Berman (1971). The GAMMA2 used in WOFOST Version 6.0 is different
from the GAMMA routine available in the TTUTIL library.
\end{indenting}
\begin{tabbing}
\hspace{1.27cm}\=\hspace{1.27cm}\=\hspace{1.27cm}\=\hspace{1.27cm}\=%
\hspace{1.27cm}\=\hspace{1.27cm}\=\hspace{1.27cm}\=\hspace{1.27cm}\=%
\hspace{1.27cm}\=\hspace{1.27cm}\=\kill
MENMET\> \> This routine provides input options for the weather data.\\
MENRAN\> \> This routine provides input options for rainfall data (daily, monthly).\\
MENU\> \> This routine provides input options for the crop, soil and weather data.
\end{tabbing}
METEO
\testlastline

\begin{indenting}{2.54cm}
This routine gets meteorological data for the current day.
\end{indenting}
NUTRIE
\testlastline

\begin{indenting}{2.54cm}
This routine calculates nutrient requirements and nutrient limited yields
according to the QUEFTS-system (Jansen et all., 1988, and Jansen and
Wolf, 1988) and generates output.
\end{indenting}
\begin{tabbing}
\hspace{1.27cm}\=\hspace{1.27cm}\=\hspace{1.27cm}\=\hspace{1.27cm}\=%
\hspace{1.27cm}\=\hspace{1.27cm}\=\hspace{1.27cm}\=\hspace{1.27cm}\=%
\hspace{1.27cm}\=\hspace{1.27cm}\=\kill
PAUZE\> \> This routine waits for $<$RETURN$>$ from key board.
\end{tabbing}
PENMAN
\testlastline

\begin{indenting}{2.54cm}
This routine calculates the potential evapo(transpi)ration rates from a free
water surface, a bare soil surface and a crop canopy in mm d$^{{\rm -1}}$. For these
calculations the analysis by Penman is followed. (Frere and Popov, 1979;
Penman, 1948, 1956 and 1963).
\end{indenting}
PRHEAD
\testlastline

\begin{indenting}{2.54cm}
This routine prints the headers for the selected kind of output files to the
output files.
\end{indenting}
PRIJRC
\testlastline

\begin{indenting}{2.54cm}
This routine writes output to a file according to the specifications for the so
called JRC-project.
\end{indenting}
PRIWFD
\testlastline

\begin{indenting}{2.54cm}
This routine writes output to a file for the water limited production run
(WATFD used as soil water balance routine).
\end{indenting}
PRIWGW
\testlastline

\begin{indenting}{2.54cm}
This routine writes output to a file for the water limited production run
(WATGW used as soil water balance routine).
\end{indenting}
PRIWPP
\testlastline

\begin{indenting}{2.54cm}
This routine writes output to a file for the potential production run (WA\-TPP used as soil water balance routine).
\end{indenting}
RANDOM
\testlastline

\begin{indenting}{2.54cm}
This function generates pseudo random numbers uniformly distributed
between 0 and 1. The algorithm used is developed by Wichman and Hill
(1982).
\end{indenting}
REPRD
\testlastline

\begin{indenting}{2.54cm}
This routine reads daily weather data. This are the so called repaired
DBMETEO daily weather data.
\end{indenting}
RNDIS
\testlastline

\begin{indenting}{2.54cm}
This routine calculates the daily rainfall by distributing the given monthly
rainfall over a defined number of randomly chosen rainy days. Monthly
rainfall and number of rainy days equal the given values. 
\end{indenting}
RNGEN
\testlastline

\begin{indenting}{2.54cm}
This routine generates one year of daily rainfall data on the basis of the
given long term average monthly rainfall and number of rainy days. The
method follows the proposal of Shu Geng et al. (1986). The procedure is a
combination of a Markov chain and a gamma distribution function.
\end{indenting}
RNREAL
\testlastline

\begin{indenting}{2.54cm}
This routine reads rainfall data from an external file called RAFILE.
Linked with TTUTIL.
\end{indenting}
ROOTD
\testlastline

\begin{indenting}{2.54cm}
In this routine the depth of the root zone is calculated for each day of the
crop cycle.
\end{indenting}
SELOUT
\testlastline

\begin{indenting}{2.54cm}
This routine enables the user to select the kind of output files he requires.
\end{indenting}
STATPP
\testlastline

\begin{indenting}{2.54cm}
Statistics of potential production. Subroutine for statistical treatment of
yield variables, simulated in a series of simulation runs with WOFOST
Version 6.0. The yield data refer to one crop and either a time series for
one weather station (daily weather), or a series over different stations or
station-years (mean monthly weather).
\end{indenting}
STATWP
\testlastline

\begin{indenting}{2.54cm}
Statistics of production, limited by availability of water. Subroutine for
statistical treatment of yield variable, simulated in a series of simulation
runs with WOFOST Version 6.0, using the soil water balance model for
free drainage. The yield data refer to one crop and either a time series for
one weather station (daily weather), or a series over different stations or
stations-years (mean monthly weather).
\end{indenting}
\begin{tabbing}
\hspace{1.27cm}\=\hspace{1.27cm}\=\hspace{1.27cm}\=\hspace{1.27cm}\=%
\hspace{1.27cm}\=\hspace{1.27cm}\=\hspace{1.27cm}\=\hspace{1.27cm}\=%
\hspace{1.27cm}\=\hspace{1.27cm}\=\kill
STDAY\> \> This routine calculates the start day if a variable sowing date is chosen.\\
STINFO\> \> Find and loads station information in the program.
\end{tabbing}
SUBSOL
\testlastline

\begin{indenting}{2.54cm}
This routine calculates the rate of capillary flow or percolation between
ground water table and root zone. The stationary flow is found by Gaussian
integration. In an iteration loop the correct flow is found. The integration
goes at most over four intervals.
\end{indenting}
SWEAF
\testlastline

\begin{indenting}{2.54cm}
The fraction of easily available soil water between field capacity and
wilting point is a function of the potential evapotranspiration rate (for a
closed canopy) in cm d$^{{\rm -1}}$ and the crop group number (from 1 {\nobreak}(=drought-sensitive) to 5 (=drought-resistent)). This function describes this relation\-ship given in tabular form by Doorenbos \& Kassam (1979) and by Van
Keulen \& Wolf (1986).
\end{indenting}
TOTASS
\testlastline

\begin{indenting}{2.54cm}
This routine calculates the daily gross CO$_{{\rm 2}}$ assimilation by performing a
Gaussian integration over time. At three different times of the day, irra\-diance is computed and used to calculate the instantaneous canopy assimila\-tion, whereafter integration takes place. 
\end{indenting}
W60MAIN
\testlastline

\begin{indenting}{2.54cm}
This routine is the driving routine for running the WOFOST Version 6.0
crop growth simulation model. The FSE system for reading data has been
applied for the standard crop data, standard physical soil data, location-specific data on agro-hydrological conditions and run specific information
on crop-calendar, weather data specification and kind of output. All data
can be varied across successive model runs by specify\-ing them in a specific
file called RERUNS.DAT.
\end{indenting}
WATFD
\testlastline

\begin{indenting}{2.54cm}
This subroutine keeps track of the soil water balance for freely draining
soils in the water limited production situation.
\end{indenting}
WATGW
\testlastline

\begin{indenting}{2.54cm}
In this routine the simulation of soil water balance is performed for soils
influenced by the presence of groundwater. Two situations are distin\-guished: with or without artificial drainage. The soil water balance is
calculated for a cropped field in the water-limited production situa\-tion.
\end{indenting}
WATPP
\testlastline

\begin{indenting}{2.54cm}
In this routine the variables of the soil water balance in the potential
production situation are calculated. The purpose is to quantify the crop
requirements for continuous growth without drought stress. It is as\-sumed
that the soil is permanently at field capacity.
\end{indenting}
WEATHR
\testlastline

\begin{indenting}{2.54cm}
This routine reads the daily weather data. The format developed by AB-DLO is used.
\end{indenting}
WOF60
\testlastline

\begin{indenting}{2.54cm}
Driving routine for the simulation routine WOFSIM. Takes care of run
control, input/output options, the call of the potential and water limited run
and the call of the years specific calculations.
\end{indenting}
WOFSIM
\testlastline

\begin{indenting}{2.54cm}
This routine organizes the simulation of crop growth and of soil water
balance conditions, the generation of weather data and the writing of
reports by successive calls to the relevant weather, crop, soil water and
printing subroutines in that order.
\end{indenting}

\bigskip
\bigskip
\bigskip
\bigskip
{\bf {\large Subroutines and functions from FORTRAN utility library TT\-UTIL}}

AMBUSY
\testlastline

\begin{indenting}{2.54cm}
This routine stores and returns codes belonging to module names to find
out whether or not other subroutines have been called.
\end{indenting}
\begin{tabbing}
\hspace{1.27cm}\=\hspace{1.27cm}\=\hspace{1.27cm}\=\hspace{1.27cm}\=%
\hspace{1.27cm}\=\hspace{1.27cm}\=\hspace{1.27cm}\=\hspace{1.27cm}\=%
\hspace{1.27cm}\=\hspace{1.27cm}\=\kill
DECCHK\> \> Checks if a certain string is a number.\\
DECINT\> \> Decodes an integer number from a character string.\\
DECREA\> \> Decodes a real number from a character string.
\end{tabbing}
ENTDCH
\testlastline

\begin{indenting}{2.54cm}
Interactive entry of a character string with a default. Writes a text on the
screen as a "question" and returns entered string to calling program.
"Question" can be specified.
\end{indenting}
ENTDIN
\testlastline

\begin{indenting}{2.54cm}
Interactive entry of an integer number with a default. Writes a text on the
screen as a "question" and returns entered number to calling program.
"Question" can be specified.
\end{indenting}

\bigskip
ENTDRE
\testlastline

\begin{indenting}{2.54cm}
Interactive entry of an real number with a default. Writes a text on the
screen as a "question" and returns entered number to calling program.
"Question" can be specified.
\end{indenting}
ERROR
\testlastline

\begin{indenting}{2.54cm}
Writes an error message to the screen and holds the screen until $<$RE\-TURN$>$ is pressed.
\end{indenting}
EXTENS
\testlastline

\begin{indenting}{2.54cm}
Changes extension of filename. Output filename is filled with characters of
input filename and new extension until end is reached. Output filen\-ame is
in uppercase characters. The old extension is the part of the filename that
follows a dot (.). A dot before a bracket (]) is neglected (VAX). The input
filename does not necessarily have an extension.
\end{indenting}
GETREC
\testlastline

\begin{indenting}{2.54cm}
Reads records from an open file skipping comment lines. Comment lines
have an asterisk (*) in their first or second column (with a space in the
first).
\end{indenting}
IFINDC
\testlastline

\begin{indenting}{2.54cm}
Finds number of name in a list with names. When name is not in the list a
zero value is returned. Character strings should be of the same length.
\end{indenting}
ILEN
\testlastline

\begin{indenting}{2.54cm}
Determines the significant length of a string. If the string is empty a zero is
returned.
\end{indenting}
ISTART
\testlastline

\begin{indenting}{2.54cm}
Determines the first significant character of a string. If the string cont\-ains
no characters, a zero value is returned.
\end{indenting}
LIMIT
\testlastline

\begin{indenting}{2.54cm}
Returns value of X limited within a specified interval.
\end{indenting}
LOWERC
\testlastline

\begin{indenting}{2.54cm}
Converts a character string to lowercase characters.
\end{indenting}
MOFLIP
\testlastline

\begin{indenting}{2.54cm}
Moves the file pointer across comment lines of data files and puts the file
pointer at the first non-comment record. Comment lines have an asterisk
(*) in their first or second column (with a space in the first).\hfill  
\end{indenting}
RDAREA
\testlastline

\begin{indenting}{2.54cm}
Reads an array of real values from a data file, that should be initialized
with RDINIT.
\end{indenting}
RDDATA
\testlastline

\begin{indenting}{2.54cm}
This is the central subroutine of the complete set of RD* routines in the
library TTUTIL. The simple routines RDSINT, RDSREA, RDAR\-EA,
RDSETS and RDFROM form user interfaces. The actual work is done in
this routine. That implies that the headers of all user interfaces also apply
to this routine RDDATA. The ITASK values are:
\end{indenting}
\begin{tabbing}
\hspace{1.27cm}\=\hspace{1.27cm}\=\hspace{1.27cm}\=\hspace{1.27cm}\=%
\hspace{1.27cm}\=\hspace{1.27cm}\=\hspace{1.27cm}\=\hspace{1.27cm}\=%
\hspace{1.27cm}\=\hspace{1.27cm}\=\kill
\>\> ITASK = 1 $<$---- RDSETS\\
\>\> ITASK = 2 $<$---- RDFROM\\
\>\> ITASK = 3 $<$---- RDINIT\\
\>\> ITASK = 4 $<$---- RDSINT, RDSREA and RDAREA.
\end{tabbing}

\zerotestlastline
\begin{indenting}{2.54cm}
Rerun and data files both are analyzed by routine RDINDX called for
ITASK = 1 and 3.
\end{indenting}
RDINDX
\testlastline

\begin{indenting}{2.54cm}
Produces an index of a data file. The index consists of a list of variable
names and an integer array pointing to decode values on a direct access
file. The direct access file is opened for reading at the end of the routine or
deleted after errors.
\end{indenting}
RDINIT
\testlastline

\begin{indenting}{2.54cm}
Initializes data file reading with the subroutines RDSINT, RDSREA and
RDAREA. An index of the data file is stored as a local array containing
variable names. The values are written to a temporary file. After a call to
RDINIT the data file itself is closed again.
\end{indenting}
\begin{tabbing}
\hspace{1.27cm}\=\hspace{1.27cm}\=\hspace{1.27cm}\=\hspace{1.27cm}\=%
\hspace{1.27cm}\=\hspace{1.27cm}\=\hspace{1.27cm}\=\hspace{1.27cm}\=%
\hspace{1.27cm}\=\hspace{1.27cm}\=\kill
RDSCHA\> \> Reads a single character value from a data file.
\end{tabbing}

\bigskip
RDSETS
\testlastline

\begin{indenting}{2.54cm}
Initializes subroutine RDDATA for reading data from a so called "rerun
file". It contains sets of variable names with associated values. The sets are
used to replace corresponding data items in a normal data file analyzed
with the routines RDINIT, RDSREA, RDS\-INT and RDAREA. This facility
has a "global" character. A call to RDINIT does not disable the replace\-ment of values. So the sets in a rerun file may contain variable names
occurring in different data files. Only a call to RDSETS with empty
filename will deactivate the re\-placement of numbers. After a normal
RDSETS call the 0-th set is activated meaning that data file values are
used. With RDFROM the sets of the rerun are actually activated.
\end{indenting}
RDSINT
\testlastline

\begin{indenting}{2.54cm}
Reads a single integer value from a data file. The reading should be
initialized with RDINIT.
\end{indenting}
RDSREA
\testlastline

\begin{indenting}{2.54cm}
Reads a single real value from a data file. The reading should be initialized
with RDINIT.
\end{indenting}
TIMER
\testlastline

\begin{indenting}{2.54cm}
This subroutine updates the time from the start of the simulation and
related variables each time it is called with ITASK=2. It will set a flag to
.TRUE. if the finish time of the simulation is reached (counted from start
of simulation). The routine is initialized first by a call with ITASK\-=1. The
first six arguments will then be made local. Leap years are handled correct\-ly.
\end{indenting}
UPPERC
\testlastline

\begin{indenting}{2.54cm}
Converts a character string to uppercase characters.
\end{indenting}
WORDS
\testlastline

\begin{indenting}{2.54cm}
Returns position and start and end of the (first) number of words to be
found in a character string. Valid separators are all the characters present
in a string containing the separator characters. 
\end{indenting}

